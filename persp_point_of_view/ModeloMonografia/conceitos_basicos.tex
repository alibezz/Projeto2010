\chapter{Técnicas básicas e ferramentas utilizadas}

\textbf{Introduzir o capítulo quando tudo já estiver escrito.}
%Neste capítulo, serão descritas técnicas básicas de Aprendizado de Máquina e Processamento de Linguagem Natural, importantes para a compreensão dos capítulos seguintes desta monografia. Na seção \textbf{Classificadores}, serão discutidos os funcionamentos dos classificadores Naïve Bayes e \emph{Support Vector Machines} (SVMs), comuns em métodos de Mineração de Perspectiva baseados em frequências de palavras. Na seção \textbf{Modelos Gráficos}, serão discutidas a notação e hipóteses deste tipo de modelo, os \textbf{Modelos Generativos}, uma de suas subcategorias bastante explorada nesta monografia, e dois modelos generativos que, a partir da interpretação de documentos como uma mistura de tópicos, geram agrupamentos de palavras que facilitam a compreensão segmentada do conteúdo de um corpus. Estes modelos são o \emph{Latent Dirichlet Allocation} (LDA) e o \emph{Labeled-Latent Dirichlet Allocation} (L-LDA), utilizados em experimentos conduzidos ao longo de todo o projeto.

%\textbf{serão discutidos os modelos de tópicos \emph{Latent Dirichlet Allocation} (\textbf{LDA}) e \emph{Labeled Latent Dirichlet Allocation} (\textbf{L-LDA}). O segundo, que consiste em uma pequena modificação do primeiro, é utilizado em parte dos experimentos conduzidos neste projeto.}


\section{Classificador Naïve Bayes}
\label{subsection:naive}

%Explicar q ele é mto utilizado nos trabalhos pra classificar perspectiva, apresentando o bom desempenho pelo qual ficou conhecido na classificaçẽo de documentos por assunto.
%Falar que ele assume q as palavras num documento são independentes. E q a ordem das palavras n importa


O Naïve Bayes é um classificador que se baseia no Teorema de Bayes e assume a \textbf{independência condicional} entre as palavras\footnote{Outros elementos dos documentos podem ser considerados, como sequências de \ensuremath{n} palavras (\ensuremath{n}-gramas).} contidas nos documentos \cite{naive-forty}. Isso significa que, para o Naïve Bayes, as palavras em qualquer documento ocorrem independentemente umas das outras. Além disso, o classificador desconsidera a ordem das palavras nos textos: \emph{casa de aline} e \emph{aline casa de} são interpretados da mesma forma. Apesar dessas suposições simplificarem bastante a estrutura linguística dos documentos, o Naïve Bayes foi um dos classificadores mais explorados nos trabalhos revisados para essa monografia. \textbf{Sete de treze} estudos, voltados para classificação, fazem uso dele em pelo menos uma parte de seus experimentos. 

% Mostrar o Teorema de Bayes e falar da prior dos parametros, mostrando a cara das coisas em si

Dado um documento \ensuremath{d_i} pertencente a um corpus \ensuremath{D} com um vocabulário \ensuremath{V}, o Naïve Bayes deve buscar a classe \ensuremath{c_j}, pertencente ao conjunto de classes \ensuremath{C}, que maximize a seguinte aplicação do Teorema de Bayes \cite{naive-forty}

\begin{equation}
\label{teorema-bayes}
\ensuremath{P(c_j | d_i) = \frac{P(c_j)P(d_i | c_j)}{P(d_i)}}
\end{equation}


% Falar q isso daí tem tudo a ver com contagem de palavras, mas tb pode ser um pouco diferente.
% Falar q normalmente usa uma técnica pra aproximar isso da distribuição real

Na prática, como esse é um problema de maximização e a probabilidade \ensuremath{P(d_i)} independe de qualquer classe, ela pode ser abstraída. Considerando-se o lado direito da Equação \ref{teorema-bayes}, tem-se que a probabilidade de se selecionar uma classe \ensuremath{c_j} qualquer é dada por uma distribuição binomial \cite{resnik}

\begin{equation}
\label{binomial}
\ensuremath{c_j \sim binomial(|C|, \pi)  \equiv  \dbinom{|C|}{c_j}\pi^{c_j}(1 - \pi)^{|C| - c_j}}
\end{equation}

onde \ensuremath{\pi} é um parâmetro da distribuição binomial. Ele é amostrado de acordo com uma distribuição beta, definida por dois parâmetros \ensuremath{\alpha} e \ensuremath{\beta}, cujos valores devem ser fixados antes de se iniciar o processo de classificação \cite{resnik}

\begin{equation}
\label{beta}
\ensuremath{\pi \sim beta(\alpha, \beta) \equiv \frac{1}{B(\alpha, \beta)}\pi^{\alpha - 1}(1 - \pi)^{\beta - 1}}
\end{equation}

A função \ensuremath{B} é aplicada aos valores \ensuremath{\alpha} e \ensuremath{\beta} para garantir que a distribuição de probabilidade beta, quando integrada, some um \cite{stat-distribs}\footnote{Toda distribuição de probabilidade, quando integrada, deve totalizar exatamente um. \cite{stat-distribs}}. Como essa distribuição é utilizada para modelar o parâmetro \ensuremath{\pi} \ensuremath{antes} de serem feitas observações sobre as classes e palavras de \ensuremath{D}, ela é uma distribuição \textbf{\emph{a priori}} e \ensuremath{\alpha} e \ensuremath{\beta} recebem o nome de \textbf{hiperparâmetros} \cite{bishop}. Uma estratégia comum é escolher um mesmo valor para os dois hiperparâmetros, favorecendo uma distribuição uniforme para \ensuremath{\pi} \cite{nigam}. %nome de hiperp., diferenciando-se dos demais parâmetros. %A distribuição \ensuremath{beta} está associada ao parâmetro \ensuremath{\pi}, e não considera  Os parâmetros \ensuremath{\alpha} e \ensuremath{\beta} devem ter seus valores escolhidos antes do processo de classificação. 

Amostrado um valor para \ensuremath{c_j}, e considerando-se o lado direito da Equação \ref{teorema-bayes}, tem-se que \ensuremath{d_i} pode ser amostrado de acordo com uma distribuição multinomial \cite{resnik}

\begin{equation}
\label{multinomial}
\ensuremath{d_i \sim Multinomial(V, \theta_j) \equiv \prod_{k = 1}^{V}\theta_{j,k}^{N(w_k, d_i)}}
\end{equation}

onde \ensuremath{\theta_j} é um parâmetro da distribuição multinomial definido \textbf{especificamente} para a classe \ensuremath{c_j}. Isso significa que há \ensuremath{|C|} parâmetros \ensuremath{\theta}, um para cada classe. \ensuremath{N(w_k, d_i)}, por sua vez, é o número de vezes que a \ensuremath{k}-ésima palavra do vocabulário \ensuremath{V}, \ensuremath{w_k}, ocorre no documento \ensuremath{d_i} \cite{resnik}\footnote{É possível utilizar, alternativamente, o número de vezes que uma determinada sequência de palavras ocorre, ou ainda um \emph{bit}, representando a ausência (0) ou presença (1) da \ensuremath{k}-ésima palavra de \ensuremath{V} em \ensuremath{d_i}.}. A esse número, dá-se o nome de \textbf{contagem} de \ensuremath{w_k} em \ensuremath{d_i} \cite{nigam}. Todos os parâmetros \ensuremath{\theta}, em particular \ensuremath{\theta_j}, são amostrados por uma distribuição dirichlet definida através de hiperparâmetros \ensuremath{\gamma} - também definidos antes do processo de classificação \cite{resnik}


\begin{equation}
\label{dirichlet}
  \ensuremath{\theta_j \sim Dirichlet(\gamma_j)}
\end{equation}

Como o Naïve Bayes assume que as palavras de \ensuremath{d_i}, \ensuremath{\langle w_{d_{i,1}}, ..., w_{d_{i,|d_i|}} \rangle}, são condicionalmente independentes, a Equação \ref{teorema-bayes} pode ser reescrita como \cite{naive-forty}

\begin{equation}
\label{teorema-bayes2}
\ensuremath{P(c_j | d_i) = \frac{P(c_j)\prod_{k = 1}^{|d_i|}P(w_{d_{i,k}} | c_j)}{P(d_i)}}
\end{equation}

Neste caso, a amostragem de cada palavra \ensuremath{w_{d_{i,k}}} corresponde à \ensuremath{k}-ésima posição da amostragem de \ensuremath{d_i}, via Equação \ref{multinomial}.

Em um cenário de classificação, os valores de todos os parâmetros, classes e documentos devem ser reamostrados iterativamente, a fim de se aproximarem cada vez mais das reais distribuições contidas no corpus. Isso pode ser feito através de alguma técnica de amostragem, como \textbf{Gibbs Sampling} ou \textbf{Expectation-Maximization}. Por questões de escopo, elas não serão apresentadas nessa monografia, podendo ser consultadas no livro de Aprendizado de Máquina de Christopher Bishop \cite{bishop}. De todo modo, é importante saber que essas reamostragens consideram as contagens de palavras de \textbf{todos os documentos}. Intuitivamente, a ideia por trás da classificação de um documento é buscar a classe cujo \emph{perfil} das contagens mais se aproxime do seu \emph{perfil} de contagens. % do documento.

%, inferindo a classe de cada um deles de acordo com quais contagens são maiores para cada classe\footnote{Ou seja, de acordo com quais contagens são maiores considerando todos os documentos que foram classificados, na iteração anterior, como pertencentes a cada classe.} e quais são maiores no documento em si

Na prática, para melhorar o desempenho da classificação, cria-se um \emph{perfil inicial} de contagens para cada classe, informando-se ao Naïve Bayes a classe verdadeira de alguns documentos. Ao conjunto desses documentos, dá-se o nome de \textbf{conjunto de treinamento} \cite{bishop}. O Naïve Bayes não precisa reamostrar as classes dos documentos desse conjunto, pois elas são informadas antes da classificação. Aos documentos cujas classes não são conhecidas, dá-se o nome de \textbf{conjunto de teste} \cite{bishop}. A classificação em si, apresentada no parágrafo anterior, só se aplica a esse segundo conjunto, que pode corresponder a \ensuremath{D} ou a parte dele, caso parte de seus documentos façam parte do conjunto de treinamento.

Para esse projeto, o Naïve Bayes implementado\footnote{A implementação está disponível no repositório \emph{online} de Aline Bessa: http://github.com/alibezz.} aplica a técnica de Gibbs Sampling, amostrando valores para as classes e documentos do conjunto de teste até a classificação estabilizar. O número de iterações, fixado em 500, se mostrou mais do que suficiente para estabilizar a classificação em todos os experimentos conduzidos, apresentados nos Capítulos \ref{chap3} e \ref{estudo}. 

% de acordo com as contagens de palavras de todos os documentos \ensuremath{\pi} e \ensuremath{\theta}
%vc fala que gibbs sampling é um algoritmo iterativo de inferência
%e que a cada iteração ele escolhe novos valores para as variáveis e parâmetros que aproximam mais a distribuição real
%reamostrando a atribuição de classes pra cada documento de acordo com as contagens dos outros documentos, et
 
 % é uma constante de normalização, provenientecuja finalidade é 

\section{Classificadores SVM}
\label{subsection:SVMs}

%Falar q os docs podem ser repr. onde cada entrada é uma word count, ou tantas outras coisas.
%Falar no fim pq n usa, pq n generaliza tão bem.
SVMs são uma família de métodos que utilizam uma abordagem geométrica para classificação. Eles são fundamentalmente utilizados em problemas de classificação envolvendo duas classes, mas podem ser adaptados para problemas mais complexos. Nesta seção, serão apresentados apenas os princípios de funcionamento de SVMs para duas classes, mais comuns na literatura. Para um aprofundamento sobre SVMs aplicados a problemas com mais de duas classes, recomenda-se a leitura do livro de Aprendizado de Máquina de Christopher Bishop \cite{bishop}.

Dado um conjunto de documentos \ensuremath{D} e um conjunto \ensuremath{M} de elementos\footnote{Esses elementos podem ser, por exemplo, o vocabulário do corpus \ensuremath{D}, como no Naïve Bayes padrão apresentado na seção \ref{subsection:naive}.} de \ensuremath{D}, tem-se que cada documento \ensuremath{d \in D} é representado como um vetor \ensuremath{x \in \mathbb{R}^{|M|}}. Cada entrada de \ensuremath{x} contém um valor associado a um dos elementos de \ensuremath{M}. Se \ensuremath{M} corresponde ao vocabulário de \ensuremath{D}, por exemplo, cada entrada de \ensuremath{x} pode corresponder à \textbf{contagem} de uma palavra de \ensuremath{M} em \ensuremath{d}. Sem perda de generalidade, assume-se que \ensuremath{D} divide-se em dois conjuntos: \textbf{treinamento} e \textbf{teste}, de forma semelhante ao apresentado na seção \ref{subsection:naive}. Ainda neste sentido, assume-se também que a classe de cada documento é um inteiro: \ensuremath{1} ou \ensuremath{-1} \cite{mono-puc}. Um SVM deve, portanto, utilizar as representações do conjunto de treinamento em \ensuremath{\mathbb{R}^{|M|}} para construir os hiperplanos \ensuremath{\theta_1} e \ensuremath{\theta_{-1}}, conforme as Equações \ref{theta1:svm} e \ref{thetamenos1:svm} \cite{mono-puc}

\begin{equation}
\label{theta1:svm}
\ensuremath{\theta_1 \equiv } \textbf{x} \ensuremath{\cdot} \textbf{w} + \ensuremath{b} = 1
\end{equation}

\begin{equation}
\label{thetamenos1:svm}
\ensuremath{\theta_{-1} \equiv } \textbf{x} \ensuremath{\cdot} \textbf{w} + \ensuremath{b} = -1
\end{equation}

O objetivo inicial de um classificador SVM é escolher os parâmetros \textbf{w} e \ensuremath{b} que maximizem a distância entre esses hiperplanos. Eles podem ser definidos minimizando-se o valor de \cite{mono-puc}

%Os vetores \ensuremath{x} que incidem em algum deles recebem o nome de \textbf{vetores de suporte}\footnote{Do inglês \emph{support vectors}.} \cite{durant-smith}.

\begin{equation}
\label{optim:svm}
\ensuremath{\frac{1}{2}||}\textbf{w}\ensuremath{||^2}
\end{equation}

Para realizar essa otimização mais facilmente, o problema pode ser remodelado com multiplicadores de Lagrange \ensuremath{\{\alpha_i\}}, \ensuremath{1 \leq i \leq n}, levando à Equação \ref{lagrange}. Busca-se, então, a minimização desta equação com relação a \textbf{w} e \ensuremath{b} e maximização com relação a \ensuremath{\{\alpha_i\}}, com todo \ensuremath{\alpha_i \geq 0} \cite{mono-puc}

\begin{equation}
\label{lagrange}
\ensuremath{L(\alpha, b,}\textbf{w}\ensuremath{) = \frac{1}{2} ||}\textbf{w}\ensuremath{||^2 - \sum_{i = 1}^n \alpha_i\big[y_i(x_i \cdot}\textbf{w}\ensuremath{+ b) -1 \big]} %x_i \cdot} \textbf{w} + \ensuremath{b} = 1 para \ensuremath{y_i = 1}
\end{equation}

onde \ensuremath{n} é a cardinalidade do conjunto de treinamento. Após a obtenção dos valores de \ensuremath{\{\alpha_i\}} que maximizam a Equação \ref{lagrange}, a obtenção da classe \ensuremath{y = 1} ou \ensuremath{y = -1} de um documento do conjunto de teste, representado por um vetor \ensuremath{x_j}, é dada pelo sinal do somatório \cite{mono-puc}

\begin{equation}
\label{result:svm}
\ensuremath{y(x_j) = sinal\bigg(\sum_{i = 1}^n \alpha_iy_i(x_i \cdot x_j) + b\bigg)} %x_i \cdot} \textbf{w} + \ensuremath{b} = 1 para \ensuremath{y_i = 1}
\end{equation}

Esta solução funciona em casos nos quais as representações do conjunto de treinamento, \ensuremath{\{x_1, ..., x_n\}}, são linearmente separáveis - ou seja, obedecem à restrição \cite{mono-puc}

\begin{equation}
\label{restr2:svm}
\ensuremath{y_i(x_i \cdot} \textbf{w} + \ensuremath{b -1) \geq 0,\ i = 1,...,n}
\end{equation}

Quando esses pontos não são linearmente separáveis, essa metodologia precisa ser ajustada, modelando a classificação errônea de documentos. Isto envolve a introdução de \ensuremath{n} variáveis frouxas\footnote{Do inglês \emph{slack variables}.} \ensuremath{\epsilon_i}, uma para cada ponto \ensuremath{(x_i, y_i)}. \ensuremath{\epsilon_i = 0} se \ensuremath{y(x_i) = y_i} e \ensuremath{\epsilon_i = |y_i - y(x_i)|} em caso contrário \cite{bishop}. O SVM deve, neste caso, minimizar \cite{mono-puc}

\begin{equation}
\label{nonlin:svm}
\ensuremath{C\sum_{i=1}^n\epsilon_i + \frac{1}{2}||}\textbf{w}\ensuremath{||^2}
\end{equation}

onde \ensuremath{C} é um parâmetro responsável por controlar o compromisso entre a penalidade das variáveis frouxas e a distância máxima dos hiperplanos \ensuremath{\theta_1} e \ensuremath{\theta_{-1}} ao hiperplano \ensuremath{\theta_0} \cite{mono-puc}. Na modelagem com multiplicadores de Lagrange, a Equação \ref{lagrange} deve ser otimizada de tal forma que todo \ensuremath{\alpha_i} deve ser maximizado obedecendo à restrição \ensuremath{0 \leq \alpha_i \leq C}. Desta forma, também se obtém a Equação \ref{result:svm} para determinação da classe de um novo documento.

De acordo com um estudo de Ng e Jordan, o Naïve Bayes atinge bom desempenho com um conjunto de treinamento menor do que aquele requerido pelos SVMs \cite{ng-jordan}. Considerando essa observação e o fato de que alguns dos \emph{datasets} estudados nesse projeto não são muito grandes, todos os experimentos desenvolvidos nos Capítulos \ref{chap3} e \ref{estudo} envolvem apenas o Naïve Bayes. Apesar disso, SVMs fazem parte da metodologia de boa parte dos trabalhos revisados no Capítulo \ref{outros} e nos \textbf{ANEXOS BLA e BLI} - por esse motivo, fez-se necessário apresentá-los nessa seção.

%SVMs não foram utilizados em nenhum experimento deste projeto, mas fazem parte da metodologia de alguns dos trabalhos revisados.

 %Como um dos \emph{datasets} estudados no Capítulo \ref{chap3} e o corpus criado para o estudo de caso do Capítulo \ref{estudo} não contêm um número muito grande de documentos


\section{Métricas para mensurar o desempenho dos classificadores}

A classificação de documentos de texto de acordo com suas perspectivas é um dos principais objetivos dos trabalhos revisados neste projeto. Grande parte deles utiliza os classificadores Naïve Bayes ou \emph{Support Vector Machines} (SVMs), apresentados respectivamente nas seções \ref{subsection:naive} e \ref{subsection:SVMs}, como parte de suas metodologias. O desempenho destes classificadores é comumente medido através das seguintes métricas: \textbf{taxa de acerto}, \textbf{precisão}, \textbf{rechamada} ou \textbf{métrica F1}.

A taxa de acerto é definida pela razão entre o número de documentos classificados corretamente e todos os documentos avaliados. A precisão, medida para uma classe \ensuremath{c} qualquer, é definida pela razão entre o número de documentos classificados corretamente como \ensuremath{c} e todos os documentos classificados como \ensuremath{c}. A rechamada, medida também para uma classe \ensuremath{c} qualquer, é definida pela razão entre o número de documentos classificados corretamente como \ensuremath{c} e a soma deste valor com o número de documentos classificados erroneamente para todas as demais classes. A métrica F1, também medida para uma classe \ensuremath{c} qualquer, é dada por

\begin{equation}
\ensuremath{2 \times \frac{precisao \times rechamada}{precisao + rechamada}}
\end{equation} 

Estas métricas revelam aspectos diferentes do desempenho de um classificador. Por este motivo, é comum encontrar mais de uma delas sendo utilizada no mesmo contexo. Além de medirem desempenho, elas estabelecem critérios objetivos para a comparação entre métodos de classificação, como pode ser visto nos artigos de Lin et al. sobre o conflito Israel-Palestina \cite{lin-et-al2006} e de Efron sobre orientação cultural \cite{efron}. 


\section{Validação cruzada}

\section{LDA}
\label{subsection:LDA}


O modelo LDA parte da ideia de que um documento pode tratar de múltiplos tópicos, refletidos nas palavras que o compõem \cite{pnas}. Assim como no modelo Naïve Bayes, palavras podem ser geradas de acordo com distribuições multinomiais específicas. A diferença é que, no LDA, cada palavra é gerada a partir de uma mistura de tópicos; no Naïve Bayes, elas são geradas a partir de um só tópico (classe) \cite{gibbs-lingpipe}.

O LDA associa as palavras dos documentos a tópicos diferentes, com maior ou menor probabilidade, criando agrupamentos que se relacionam semanticamente. Os tópicos em um LDA são variáveis latentes, cujos significados requerem uma interpretação posterior ao processamento. Esta interpretação baseia-se nas relações semânticas entre as palavras que se associaram mais fortemente a cada um deles. %, algo relativamente subjetivo. %ser bastante levando  

Para ilustrar como as palavras evidenciam o significado de um tópico, um experimento envolvendo receitas culinárias extraídas do \emph{site} \textbf{allrecipes.com} foi executado. Apenas os ingredientes de cada receita foram considerados. Na Tabela 2.1, constam as cinco palavras mais fortemente associadas a quatro tópicos, de acordo com o LDA. % - ou seja, as cinco palavras para as quais as probabilidades obtidas com a equação \ref{eq2} foram mais altas. %que obtiveram mais alta probabilidade de acordo com \ref{eq2}.

\begin{table}[t]
\centering
\label{LDA:1}
\begin{tabular}{| l | p{7cm} | }
\hline
\textbf{Tópico} & \textbf{Palavras} \\ \hline
\textbf{Tópico 1} & beef, cheese, tomato, sauce, pepper \\ \hline
\textbf{Tópico 2} & chicken, breast, pastum, broth, tomato \\ \hline
\textbf{Tópico 3} & flour, sugar, butter, powder, egg  \\ \hline
\textbf{Tópico 4} & cream, cheese, butter, milk, cake \\ \hline
\end{tabular}
\caption{As cinco palavras mais fortemente associadas a quatro tópicos gerados por um LDA.}
\end{table}

Considerando que todas as receitas pertencem à culinária tradicional dos Estados Unidos, as palavras listadas na Tabela 2.1, e a forma como se associam em torno de cada tópico, são indicativos do bom funcionamento do LDA. O primeiro tópico pode ser interpretado como \textbf{ingredientes para          \emph{cheeseburger}}; o segundo, ao associar \emph{chicken}, \emph{pastum} e \emph{broth}, remete a receitas de sopas e caldos comuns em climas frios, podendo ser interpretado como \textbf{ingredientes para sopa}; o terceiro pode ser interpretado como \textbf{ingredientes para bolo}; o quarto, ao associar \emph{cream}, \emph{cheese} e \emph{cake}, pode ser interpretado como \textbf{ingredientes para \emph{cheesecake}}. É importante frisar que estas interpretações, apesar de subjetivas, indicam perspectivas culinárias distintas e coerentes internamente. Seria diferente de encontrar, por exemplo, um tópico fortemente associado às palavras \emph{sugar}, \emph{pepper} e \emph{potato}, dificilmente encontradas em uma mesma receita típica dos Estados Unidos.


%- ou seja, não são identificados antes ou durante o processamento -cujos significados requerem  só podem ser interpretados após o processamento do modelo, observando-se  

% que, após o processamento, sintetizam um significado associado ao


%atribuir um significado a cada um deles apenas observando as relações entre as palavras associadas com mais destaque. Para o LDA, os tópicos não são, portanto, pré-identificados antes do processamento, requerendo uma interpretação posterior e subjetiva de seus significados. %Neste sentido, o modelo é útil para identificar padrões contidos em documentos de texto.  

O modelo LDA trata cada documento pertencente a um conjunto de documentos \ensuremath{D} como uma mistura de tópicos, representada por uma distribuição de probabilidade sobre um conjunto de tópicos \ensuremath{T}. Cada tópico, por sua vez, é visto como uma mistura de palavras, representada por uma distribuição sobre todas as palavras distintas de \ensuremath{D}. Para cada documento \ensuremath{d \in D}, é fixada uma distribuição de probabilidade sobre tópicos \ensuremath{\theta_d}, condicionada a um hiperparâmetro \ensuremath{\alpha}

\begin{equation}
\label{lda:topic}
\ensuremath{\theta_d \sim Dirichlet(\alpha)}
\end{equation}

Para cada tópico \ensuremath{t \in T}, é fixada uma distribuição de probabilidade \ensuremath{\phi_t} sobre palavras, condicionada a um hiperparâmatro \ensuremath{\beta}

\begin{equation}
\label{lda:word}
\ensuremath{\phi_t \sim Dirichlet(\beta)}
\end{equation}

Em seguida, para cada uma das \ensuremath{n} palavras de \ensuremath{d}, um tópico \ensuremath{t} é escolhido, de acordo com \ensuremath{\theta_d}

\begin{equation}
\label{lda:topic-chosen}
\ensuremath{t \sim Discrete(\theta_d)}
\end{equation}

e uma palavra \ensuremath{w} é gerada de acordo com \ensuremath{\phi_t}

\begin{equation}
\label{lda:word-chosen}
\ensuremath{w \sim Discrete(\phi_t)}
\end{equation}

A distribuição conjunta do modelo LDA é dada por

\begin{equation}
\label{joint:lda}
\ensuremath{\prod_{i=1}^{|D|} \bigg\{P(\theta_i|\alpha)\bigg[\prod_{j=1}^{|W|}P(t_j|\theta_i)P(w_j|t_j,\beta)\bigg]\bigg\}}  
\end{equation}

\ensuremath{P(w_j|t_j,\beta)} reflete o quanto a palavra \ensuremath{w_j} se relaciona com o tópico \ensuremath{t_j}; \ensuremath{P(t_j|\theta_i)}, por sua vez, funciona como uma medida do quanto o tópico \ensuremath{t_j} é importante no contexto do documento \ensuremath{d_i} \cite{pnas}. Na Figura \ref{fig:lda}, tem-se o modelo gráfico do LDA correspondente à distribuição conjunta \ref{joint:lda}. 

\begin{figure}[t]
  \centering % este comando é usado para centralizar a Figura
  \includegraphics[width=6.5cm, height=3.2cm]{Latent_Dirichlet_allocation.png}\\
  \caption{Modelo gráfico LDA.} %\ref{joint:lda}.}
  \label{fig:lda}
\end{figure}

A implementação do LDA utilizada para estes experimentos está disponível no repositório \emph{online} de Alexandre Passos \cite{top-lda}, e o número de iterações para amostragem de tópicos e palavras foi fixado em 100. 


%O cálculo destas probabilidades também envolve integrais difíceis, ou mesmo impossíveis, de resolver analiticamente, o que implica no uso de técnicas de amostragem e aproximação. A amostragem de Gibbs, descrita superficialmente na seção \ref{subsection:bayes}, é uma alternativa para a geração de valores para variáveis aleatórias contidas no modelo \cite{gibbs-lingpipe}, sendo utilizada nos experimentos com LDA conduzidos neste projeto. 

%Quando um método de classificação é aplicado a um conjunto de documentos, a interpretação do resultado é imediata: cada documento estará associado a uma única classe. No caso do modelo de tópicos LDA, a interpretação do resultado é mais subjetiva. É preciso observar as palavras que se associam com maior probabilidade a cada tópico, buscando algum tipo de semelhança entre elas, para inferir seus significados. 


% o agrupamento de palavras como \emph{beef}, \emph{sauce} e \emph{cheese} em torno de um mesmo tópico indica o bom funcionamento do LDA%As palavras associadas a cada tópico siderando que as receitas pertencem à culinária tradicional dos Estados Unidos. O primeiro tópico pode ser interpretado como \textbf{Receitas com Carne}, pois associa ingredientes comumente consumidos  
%\textbf{Análise}

\subsection{L-LDA}

O L-LDA é uma variação do LDA em que se restringe o número de tópicos associados a cada documento. Ou seja, as distribuições fixadas para os tópicos de cada documento não necessariamente são sobre todos os tópicos \ensuremath{t \in T}. Além disso, os tópicos presentes em cada documento são identificados antes da execução do modelo, o que diminui a subjetividade envolvida na interpretação de seus significados após o processamento. 

Um bom exemplo para ilustrar a aplicação deste modelo envolve um \emph{blog}, em que cada \emph{post} é marcado com um conjunto específico de \emph{tags}. Se cada \emph{tag} é interpretada como um tópico, é possível informar ao L-LDA em que \emph{posts} cada uma delas está presente, processar os \emph{posts} com o modelo e saber, após o processamento, quais palavras se associam mais fortemente a cada \emph{tag}. Neste exemplo, existe um mapeamento direto entre os tópicos e as \emph{tags}, conduzindo a uma interpretação mais imediata do significado de cada agrupamento de palavras.

Experimentos com o L-LDA foram desenvolvidos ao longo deste projeto, associando cada tópico, por exemplo, a uma perspectiva a ser minerada. Após a execução do modelo, as palavras mais fortemente associadas a cada perspectiva ilustram como os assuntos discutidos são enfocados por elas. Quanto mais duas perspectivas se distanciam, mais diferentes são as palavras que se associam com destaque a cada uma delas.

O L-LDA foi discutido pela primeira vez em um artigo de Ramage et al. \cite{llda}, aplicado ao problema de atribuição de crédito em páginas do \emph{site del.icio.us}, marcadas com múltiplas \emph{tags}. O artigo parte da hipótese de que, embora um documento possa estar marcado com várias \emph{tags} diferentes, nem sempre elas se aplicam igualmente a todas as palavras nele contidas. A ideia da atribuição de crédito, portanto, consiste em associar cada palavra do documento às \emph{tags} mais apropriadas e vice-versa. 

%Uma aplicação de L-LDA, sugerida neste artigo e bastante empregada neste projeto, envolve, após a associação de palavras a tópicos (neste caso, \emph{tags}), a extração de trechos do documento que as contenham. Isto conduz a uma compreensão melhor de como o conteúdo se associa aos tópicos separadamente. 

A implementação de L-LDA utilizada neste projeto também está disponível no repositório \emph{online} de Alexandre Passos \cite{top-llda}. O número de iterações para amostragem de tópicos e palavras, em todos os experimentos, foi fixado em 100.
% do envolvendo uma ou mais palavras associadas a uma \emph{tag} particular. 
%O resultado de um método de classificação consiste em associar cada documento de um conjunto Diferentemente de métodos de classificação, cujo resultado pode ser interpretado de forma obj

%Assim como discutido na seção \ref{subsection:bayes}, 
% tópicos é fixada multinomial  \ensure\ensuremath{d} se associa a tópicos \ensuremath{t \in T} com probabilidades diferentes, o que é determinado pela escolha de uma distribuição de probabilidade sobre os tópicos. Em seguida, cada tópico é tratado como uma distribuição sobre palavras e cada palavra \ensuremath{w_i} de \ensuremath{d} se associa a eles de acordo com estas distribuições. A probabilidade da \ensuremath{i}-ésima palavra de \ensuremath{d} pode ser calculada como 

%P(wi) = sum P(wi | ti = j) P(ti = j), j de 1 a T. 

%onde ti é uma variável latente - ou seja, cujo valor não é observável, mas sim inferido - referente a um tópico, P(wi | ti = j) é a probabilidade de wi se associar ao tópico j e P(ti = j) é a probabilidade de se obter o tópico j na distribuição fixada para o documento \ensuremath{d} \cite{pnas}.

 %\textbf{Falar das receitas.}
%\textbf{Falar do uso de Gibbs Sampling para inferência}


%\subsection{L-LDA}



%Como o cenário de classificação envolve documentos de teste, cujas classes não são informadas ao classificador, uma saída para utilizar adequada deve-se inicialmente amostrar uma classe para cada % o cálculo desses valores se torna bastante complexo. É necessário, portanto, utilizar alguma técnica de amostragem para estimá-los. Para este projeto, o método escolhido foi o \textbf{Gibbs Sampling}, por ser simples de implementar e obter boas estimativas em um curto espaço de tempo. A técnica será apresentada brevemente nesta seção

%Não é possível obter os valores de \ensuremath{\phi} 
%A fim de se buscar estimativas para o conjunto de parâmetros que melhor se adaptem a \ensuremath{D}, Os valores de \ensuremath{\phi}, entretanto, não são usados diretamente na classificação. Em vez disso, %gera-se o documento baseando-se em seus hiperparâmetros e em informações extraídas do corpus, obtendo-se estimativas

%O gibbs sampling só é necessário quando você
%quer calcular os parâmetros P(tk|c) e P(c) usando, além dos documentos
%rotulados, documentos não rotulados. O fato de de repente o modelo
%virar um mixture model cria problemas de não-identificabilidade, e as
%somas e produtos que entram nas integrais complicam as coisas
%bastante.

%Esses valores podem ser estimados para documentos de \ensuremath{D''} de acordo com algum método de amostragem. Para esse projeto, o método escolhido foi o \textbf{Gibbs Sampling}, pe


%as equações \ref{classe2} e \ref{palavra} \cite{gibbs-lingpipe}. Antes de tudo, entretanto,  Por questões de escopo, suas derivações não serão apresentadas nessa monografia. 



%\begin{equation}
%\label{palavra}
%\ensuremath{P(w_t | c_j ; \theta') \eq \frac{1 + \sum_{i = 1}^{|D'|}z_{ij}N(w_t, d_i)}{\sum_{s = 1}^{|V|}\sum_{i = 1}^{|D'|}z_{ij}N(w_s, d_i)}}
%\end{equation}

%onde \ensuremath{z_{ij}} é dado pela classe \ensuremath{y_i} de \ensuremath{d_i}: 1 quando \ensuremath{y_i = c_j} e 0 em caso contrário. \ensuremath{N(w_t, d_i) é o número de vezes que a palavra \ensuremath{w_t} ocorre em \ensuremath{d_i}. A esse valor, dá-se o nome de \textbf{contagem} da palavra \ensuremath{w_i} em \ensuremath{d_i}, informação muito utilizada nos métodos de classificação de documentos por perspectiva. O Capítulo \ref{chap3}, inclusive, é totalmente dedicado à discussão do uso dessa informação para essa tarefa. A probabilidade de se obter uma classe \ensuremath{c_j}, por sua vez, é dada por%where N(wt , di ) is the count of the number of times word wt occurs in document di
%and where zij is given by the class label: 1 when yi = cj and otherwise 0.



%Para a classificação de documentos ser bem-sucedida, os valores de \ensuremath{\phi} devem modelar \ensuremath{D} da melhor forma possível. Isso significa que não basta amostrá-los de acordo com os hiperparâmetros \ensuremath{\{\alpha, \beta, \gamma_1, ..., \gamma_|D|\}} para obter boas estimativas para os dois termos no numerador de \ref{teorema-bayes}. É preciso estimá-los de acordo com informações extraídas dos próprios documentos. % dessa equação. e  Além disso, assume-se, no modelo Naïve Bayes, que as palavras dos documentos são condicionalmente independentes, de modo que a equação \ref{teorema-bayes} pode ser reescrita como

%O modelo Naïve Bayes é composto de um conjunto de parâmetros \ensuremath{\theta}, conforme discutido na seção \ref{subsection:naive}. Para identificar a classe \ensuremath{y_i} de um documento \ensuremath{d_i \in D''}, é preciso estimar inicialmente valores para esses parâmetros, gerando um novo conjunto \ensuremath{\theta'}. Embora seja possível considerar apenas o conjunto de treinamento 



%Essas estimativas devem maximizar \ensuremath{P(\theta | D')} ou \ensuremath{P(\theta | D)}. Pelo Teorema de Bayes, tem-se que, no primeiro caso, \ensuremath{P(\theta | D') \prop P(D' | \theta)P(\theta)}. Para se maximizar essa relação, deve-se resolver um sistema de derivadas parciais do \ensuremath{log(P(\theta | D')} utilizando-se uma técnica denominada Maximum a Posteriori (MAP) \cite{nigam}, tópico que não será abordado nessa monografia. É suficiente saber que essa maximização conduz às expressões \ref{palavra} e \ref{classe2}, muito importantes para a compreensão dos próximos capítulos\footnote{Essas expressões podem variar um pouco, a depender do método utilizado para derivá-las. As versões apresentadas aqui foram retiradas da dissertação de Kamal Nigam \cite{nigam}.}. A probabilidade de se obter uma palavra \ensuremath{w_t \in V}, fixada uma classe \ensuremath{c_j}, é dada por



%O segundo caso é análogo.}



%O processo de maximização, em ambos os casos, é relativamente complexo, fugindo ao escopo dessa monografia. Ele conduz, entretanto, a expressões bastante simples matematicamente, utilizadas na implementação dos classificadores%obtidas, entretanto, são bastante simples matematicamente % Basicamente, elas são razões entre % baseiam exclusivamente no número de vezes que cada palavra ocorre em um documento ou em uma classe %relacionam o número de ocorrências de cada palavra ocorre em um documento ou em uma classe% utilizadas efetivamente pelos classificadorsentretanto, são  %O primeiro caso é mais simples, pois envolve apenas documentos para os quais já se conhecem as classes. % Por este motivo, ele será discutido inicialmente.% O segundo caso requer mais informações, sendo discutido % \ensuremath{\theta'}% que maximizem \ensuremath{arg max_j P(y_i = c_j | d_i ; \theta')}. Aplicando-se o Teorema de Bayes, tem-se que
 
                                                 %This is the value of θ that is most probable given the evidence of the training data and a prior.


%\begin{equation}
%\label{palavra}
%\ensuremath{P(w_t | c_j ; \theta') \eq \frac{1 + \sum_{i = 1}^{|D'|}z_{ij}N(w_t, d_i)}{\sum_{s = 1}^{|V|}\sum_{i = 1}^{|D'|}z_{ij}N(w_s, d_i)}}
%\end{equation}

%onde \ensuremath{z_{ij}} é dado pela classe \ensuremath{y_i} de \ensuremath{d_i}: 1 quando \ensuremath{y_i = c_j} e 0 em caso contrário. \ensuremath{N(w_t, d_i) é o número de vezes que a palavra \ensuremath{w_t} ocorre em \ensuremath{d_i}. A esse valor, dá-se o nome de \textbf{contagem} da palavra \ensuremath{w_i} em \ensuremath{d_i}, informação muito utilizada nos métodos de classificação de documentos por perspectiva. O Capítulo \ref{chap3}, inclusive, é totalmente dedicado à discussão do uso dessa informação para essa tarefa. A probabilidade de se obter uma classe \ensuremath{c_j}, por sua vez, é dada por%where N(wt , di ) is the count of the number of times word wt occurs in document di
%and where zij is given by the class label: 1 when yi = cj and otherwise 0.

%\begin{equation}
%\label{classe2}
%\ensuremath{P(c_j | \theta') \eq \frac{1 + \sum_{i = 1}^{|D'|}z_{ij}}{|C| + |D'|}}
%\end{equation}

%Não é possível aplicar as equações \ref{palavra} e \ref{classe2} diretamente no segundo caso, pois o classificador não sabe que classes correspondem aos documentos de \ensuremath{D''}. Para se obter boas estimativas para \ensuremath{\theta} nesse caso, uma saída é utilizar um método de amostragem, como \ensuremath{Gibbs Sampling}. %- e, consequentemen neste caso, conduzindo  % \ensuremath{\theta'} considerando todos os documentos de \ensuremath{D}, uma saída é utilizar um método de amostragem, como Gibbs Sampling. amostrar, inicialmente, a classe de cada \ensuremath{d \in D''}, considerando as distribuições de probabilidade \ref{theta_c} e \ref{classe} apresentadas na seção \ref{subsection:naive}. Neste momento, é interessante que a escolha de qualquer classe seja equiprovável, dado que não se sabe quantos documentos de \ensuremath{D''} efetivamente pertencem a cada uma delas. Com todos os documentos rotulados, é possível obter uma estimativa para \ensuremath{P(\theta | D)}, mas nada garante que ela seja a melhor possível. O que deve ser feito, neste caso, é reclassificar os documentos de \ensuremath{D''} diversas vezes, até que se estabilizem os valores estimados para \ensuremath{\theta} - ou seja, até que o modelo convirja. Esses valores configuram uma estimativa ótima para \ensuremath{\theta}. Para reclassificar os documentos de \ensuremath{D''}

%utilizar as equações \ref{palavra} e \ref{classe2}, trocando \ensuremath{D'} por \ensuremath{D}. O que se obtêm, entretanto, não são estimativas ideais para \ensuremath{w_t} e \ensuremath{c_j}, pois os parâ


%Entretanto, essa primeira classificação não é suficiente, pois os parâmetros obtidos não necessariamente maximizam \ensuremath{P(\theta | D)}. Para se obter os melhores parâmetros que modelam \ensuremath{D} % e utilizar esses valores para 


%Dizer que isso é preferível pois modela melhor.

 %  parte dos documentos contidos em \ensuremath{D} (\ensuremath{D''}). %o segundo caso, antes de resolver a maximização que conduz a equações análogas a \ref{palavra} e \ref{classe}, é preciso estimar valores iniciais  % tamb
%the highest posterior probability, arg maxj P(yi = cj |di ; θ)


% O objetivo inicial do classificador Naïve Bayes, portanto, é obter um \ensuremath{\theta} que maximize \ensuremath{P(\theta | D')} ou \ensuremath{P(\theta | D)}. No primeiro caso, tem-se, utilizando o Teorema de Bayes, que

%\begin{equation}
%\label{gerando-theta}
%\ensuremath{P(\theta | D') \eq \frac{P(D'|\theta)P(\theta)}{P(D')} 
%\end{equation}

%A maximização dessa equação é bastante complexa, fugindo completamente do escopo dessa monografia. A expressão obtida, entretanto, é fundamental para a compreensão dos próximos capítulos


 %conjunto de valores para eles que 




%dor de documentos a partir dele. Esta seção trata de documentos de texto em particular, por se tratarem do objeto básico de estudo deste projeto. 
%Sabe-se que, em um Naïve Bayes, assume-se que as as características em um documento são condicionalmente independentes, o que equivale a afirmar, por exemplo, que a presença de uma palavra em um documento de texto não é informativa sobre a presença de nenhuma outra. Apesar desta hipótese simplificar bastante a estrutura linguística de um texto, classificadores construídos a partir do modelo Naïve Bayes, denominados classificadores Naïve Bayes, reportam um bom desempenho em várias tarefas de classificação baseadas em palavras \cite{naive-at-forty} \cite{mccallum-naive}.

%Dados um documento \ensuremath{d} pertencente a um conjunto de documentos \ensuremath{D}, todas as palavras distintas de \ensuremath{D}, \ensuremath{F_1, ..., F_k}, uma variável aleatória \ensuremath{c}, que representa as possíveis classes de \ensuremath{d}, e um vetor \ensuremath{v_d}, em que cada posição corresponde a uma de suas \ensuremath{n} palavras, tem-se que
 

%conforme discutido anteriormente na seção \ref{subsection:naive}. Um classificador Naïve Bayes deve rotular o documento \ensuremath{d} com o valor de \ensuremath{c} que maximiza a equação \ref{eq2:bayes}. Como o denominador na equação \ref{eq2:bayes} é o mesmo para todas as classes, ele pode ser ignorado nestes cálculos.

%Normalmente, classificadores Naïve Bayes são utilizados de forma semi-supervisionada. Isto significa que eles são submetidos a uma etapa de treinamento, na qual aprendem as classes associadas a alguns documentos, e a uma etapa de classificação, na qual devem simular o processo gerador destes documentos e utilizar esta informação para classificar outros. Basicamente, as informações aprendidas na etapa de treinamento modelam as distribuições das palavras de \ensuremath{D} por classe, gerando parâmetros para as distribuições de probabilidade envolvidas na classificação de outros documentos. 


%O número de iterações para amostragem de tópicos e palavras, em todos os experimentos, foi fixado em 100.

% a amostragem da equação \ref{eq:bayes} foi fixado em 500.

% classificando outros                        Dado que tenho exemplos de texto de cada classe. O
%que posso inferir sobre o processo gerador destes textos


%                              e                  o classificador reporta
%o melhor desempenho em v ́rias tarefas de classifica ̧ ̃o. Este fenˆmeno  ́
%                        a                         ca

 
%\begin{algorithm}
%\ensuremath{z^{(0)}} \gets \ensuremath{\langle z_1^{(0)}, \ldots, z_k^{(0)}\rangle}
%\FOR{\ensuremath{t = 1} to \ensuremath{T} do}
 % \FOR{\ensuremath{i = 1} to \ensuremath{k} do}
 %   \ensuremath{z_i^{(t + 1)} ~ P(Z_i | z_1^{(t + 1)}, \ldots, z_{i - 1}^{(t + 1)}, z_{i + 1}^{(t)}, \ldots, z_k^{(t)})}
 % \ENDFOR
%\ENDFOR

%\end{algorithm}


%Além disso, você deveria começar a descrever o naive bayes falando que
%ele é um modelo generativo e que assume independência condicional das
%features dado as classes. Dizer que naive bayes aproxima P(vetor(d),c)
%como sendo P(c) Produtorio(P(di|c), o que é a mesma coisa que dizer
%que as palavras são condicionalmente independentes dado as classes,
%mostrar como vc faz pra usar isso pra classificar, usando P(c|d)  =
%P(c,d)/P(d), e que como P(d) independe da classe pode ser ignorado se
%você quer encontrar a melhor classe. Daí você deveria especificar de
%onde vêm essas probabilidades, colocando uma prior Beta(alpha, alpha)
%em P(c) e uma prior Dirichlet(alpha) em P(tk|c).

%Dado um conjunto de documentos \ensuremath{D} e um conjunto de classes \ensuremath{C}, o classificador Naïve Bayes estima, através da aplicação do Teorema de Bayes,  a probabilidade de cada \ensuremath{d \in D} ser de cada uma das classes \ensuremath{c \in C}. Com estas probabilidades calculadas, o classificador determina, para todo \ensuremath{d}, qual é a classe \emph{c} a que ele estará associado. Esta classe pode, por exemplo, ser aquela para a qual a probabilidade obtida foi simplesmente a mais alta \cite{durant-smith} \cite{naive-forty}.

%Este tipo de classificador parte da ideia de que as informações presentes em um documento, utilizadas na determinação de sua classe, são independentes entre si. No caso de um documento de texto, assume-se que a presença ou ausência de um termo - uma palavra ou uma sequência de palavras - é independente da presença ou ausência de qualquer outro. Definida esta hipótese, a probabilidade de que um documento \emph{d} seja de uma classe \emph{c} é tal que

%\begin{equation}
%\label{eq1}
%\ensuremath{P(c|d) \propto P(c)\prod_{k=1}^{t_d}P(t_k|c)}  
%\end{equation} 

%onde \ensuremath{P(t_k|c)} é a probabilidade condicional do termo \ensuremath{t_k}  ocorrer em um documento da classe \ensuremath{c},  \ensuremath{P(c)} é a probabilidade a priori de um documento qualquer pertencer à classe \ensuremath{c} e \ensuremath{t_d} é o número de termos em \emph{d} \cite{stanford-IRbook}.

%As probabilidades envolvidas na relação \ref{eq1} contêm integrais díficeis ou mesmo impossíveis de se calcular analiticamente. Para calcular \ensuremath{P(c|d)}, portanto, utiliza-se aproximações obtidas através de técnicas de amostragem. Uma destas técnicas, comum na literatura de Aprendizado de Máquina e empregada neste projeto, é a amostragem de Gibbs. Em uma iteração da amostragem, a técnica condiciona as probabilidades calculadas para um documento \ensuremath{k} às classificações obtidas para os \ensuremath{k - 1} documentos anteriores \cite{resnik-gibbs}. A técnica está descrita, de forma básica, no algoritmo \textbf{X (o algorithm2e.sty tá dando pau)}.
                             % The basic idea in Gibbs sampling is that, rather than probabilistically picking
%the next state all at once, you make a separate probabilistic choice for each of the k dimensions, where each
%choice depends on the other k − 1 dimensions.


%A hipótese de independência entre os termos, razão pela qual o Naïve Bayes tem este nome\footnote{Naïve é uma palavra de origem francesa que significa "ingênua"}, simplifica bastante a estrutura da informação contida nos documentos. Ainda assim, o classificador costuma apresentar boas performances em categorização de textos, sendo utilizado, por exemplo, como base metodológica para alguns filtros de \emph{spam} \cite{paul}. Para melhorar o desempenho do Naïve Bayes, é comum fixar um conjunto de documentos previamente classificados de forma correta e utilizar a informação sobre suas classes na determinação das classes de outros documentos. Ao conjunto de documentos previamente classificados, dá-se o nome de \textbf{conjunto de treinamento}; ao conjunto de documentos a serem classificados, \textbf{conjunto de teste}.

 %Através do uso de Amostragem de Gibbs, a cada iteração se obtém uma aproximação melhor O número de amostras coletadas envolvendo cada classe foi fixado em 500. 
 
 
%     We shall assume for the moment that the training data set is linearly separable in
%feature space, so that by definition there exists at least one choice of the parameters
%w and b such that a function of the form (7.1) satisfies y(xn ) > 0 for points having
%tn = +1 and y(xn ) < 0 for points having tn = −1, so that tn y(xn ) > 0 for all
%training data points.


%Suppose some given data points each belong to one of two classes, and the goal is to decide which class a new data point will be in. In the case of support vector machines, a data point is viewed as a p-dimensional vector (a list of p numbers), and we want to know whether we can separate such points with a p − 1-dimensional hyperplane. This is called a linear classifier. There are many hyperplanes that might classify the data. One reasonable choice as the best hyperplane is the one that represents the largest separation, or margin, between the two classes. So we choose the hyperplane so that the distance from it to the nearest data point on each side is maximized. If such a hyperplane exists, it is known as the maximum-margin hyperplane and the linear classifier it defines is known as a maximum margin classifier.

% cada documento do corpus é representado como um vetor \ensuremath{x} em um espaço euclidiano \ensuremath{\mathbb{R}^M}. Na etapa de treinamento, \ensuremath{N} documentos, com suas respectivas classes \ensuremath{c_0} ou \ensuremath{c_1}, são apresentadas ao classificador, que deve aprender uma função e a classificação envolve o aprendizado de uma função que corte este hiperplano, dividindo os pontos em duas classes distintas. 




