\appendix

\chapter{Teorema de Bayes e notações}

A Estatística se alia à Teoria da Probabilidade para estimar e analisar a ocorrência de eventos, como \emph{Ganho de cem reais em sorteio} ou \emph{Chuva ao meio-dia}. Dados dois eventos \emph{A} e \emph{B}, temos que a probabilidade \emph{a priori} de \emph{A} acontecer \textbf{ignora} a ocorrência do evento \emph{B}. Essa probabilidade é representada pela notação \ensuremath{P(A)} \cite{spiegelhalter}. Analogamente, para o evento \emph{B}, tem-se \ensuremath{P(B)}. Na prática sabe-se, intuitivamente, que a ocorrência de determinados eventos interfere no acontecimento de outros. Se às onze e meia da manhã observa-se o evento \emph{Céu nublado}, a probabilidade do evento \emph{Chuva ao meio-dia} ocorrer pode diferir daquela que ignora esse primeiro evento. Neste sentido, a probabilidade de um evento \emph{A} ocorrer \emph{dado} que \emph{B} ocorreu recebe o nome de probabilidade condicional de \emph{A} dado \emph{B}, e é denotada por \ensuremath{P(A\mbox{ }|\mbox{ }B)} \cite{spiegelhalter}. Analogamente, tem-se \ensuremath{P(B\mbox{ }|\mbox{ }A)}. O Teorema de Bayes correlaciona essas probabilidades da seguinte forma \cite{spiegelhalter}

\begin{equation}
\label{bayes-apendice}
\ensuremath{P(A\mbox{ }|\mbox{ }B) = \frac{P(B\mbox{ }|\mbox{ }A)P(A)}{P(B)}} 
\end{equation}

No contexto da Equação \ref{bayes-apendice}, \ensuremath{P(A\mbox{ }|\mbox{ }B)} recebe o nome de probabilidade \emph{a posteriori} de \emph{A} e \ensuremath{P(B\mbox{ }|\mbox{ }A)} recebe o nome de \emph{likelihood} \cite{spiegelhalter}. O classificador Naïve Bayes se baseia em uma aplicação direta do Teorema de Bayes.  Dados os eventos \emph{Obtenção de um documento x} e \emph{Obtenção de uma classe y}, o classificador deve estimar a ocorrência do segundo evento assumindo que o primeiro já ocorreu.   

