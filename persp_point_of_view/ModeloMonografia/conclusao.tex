\chapter{Conclusão}
\label{conclusoes}

Com a disseminação da Web, a elaboração e disseminação de textos carregados com um ponto de vista se popularizou. Não se tratam de documentos que trazem opiniões pontuais a respeito de um único objeto, como um filme ou um livro, mas sim a exposição de argumentos e ideias que, unidos, transmitem a defesa de uma posição a respeito de um certo tema. A leitura da \emph{survey} de Pang e Lee e de alguns artigos, como o de Pang, Lee e Vaithyanathan sobre filmes no IMDb\footnote{http://imdb.com/} ou o de Dave, Lawrence e Pennock sobre produtos vendidos na Amazon\footnote{http://amazon.com/}, sugere que a classificação por opiniões tem sido mais explorada em \emph{datasets} que envolvem produtos ou serviços \cite{omsa} \cite{thumbs-up} \cite{peanut-gallery}. Os artigos revisados no Capítulo \ref{chap3}, por sua vez, indicam que a classificação por perspectivas é mais aplicada a \emph{datasets} que envolvem posicionamentos políticos e temáticas polêmicas, como pena de morte.

A revisão apresentada no Capítulo \ref{chap3}, além de indicar preferências temáticas na área, também aponta algumas predileções metodológicas. A maioria dos trabalhos classifica os documentos com um Naïve Bayes ou um SVM\footnote{Esses classificadores são apresentados no Capítulo \ref{basicos}.}. Em boa parte dos casos, essa classificação explora apenas as diferentes contagens de palavras nos documentos. Alguns outros artigos exploram propriedades sintáticas ou semânticas dos \emph{datasets} analisados. Outros, por fim, consideram as interações entre os documentos na determinação de seus pontos de vista. A contagem de palavras, simples de ser computada em qualquer língua, foi a característica mais estudada pelos trabalhos revisados - ainda que não tenha recebido destaque em todos eles. Em boa parte dos casos, seu uso exclusivo já é suficiente para uma boa classificação. Em alguns outros, o uso de outras estratégias, mais complexas, foi fundamental para a melhoria na classificação. Diante disso, recomenda-se, inicialmente, para o problema de classificação por perspectiva, o uso \emph{exclusivo} de contagens de palavras. Se os resultados obtidos não forem satisfatórios, recomenda-se a escolha de um subconjunto ótimo de palavras, como indica o trabalho de Durant e Smith \cite{durant-smith}. Apenas se não houver melhorias significativas, recomenda-se o uso de outras características, um pré-processamento de documentos mais refinado ou a investigação das interações entre os documentos. Essas recomendações, portanto, partem das ideias mais simples para as mais complexas. Dessa forma, evitam-se esforços desnecessários. 
%INTERNACIONALIZÁVEEEEEL!!
%o uso de um pré-processamento mais refinado dos documentos ou 

A contagem de palavras foi explorada por muitos artigos, como característica diferenciadora de perspectivas. De fato, a classificação baseada em contagens de palavras, ou em alguma de suas variações (como a presença/ausência de palavras), assume uma hipótese defendida em estudos de Linguística de corpus: a quantidade de vezes que uma palavra é mencionada em um documento (sua contagem) está diretamente relacionada a quais ideias ele destaca \cite{teubert}. Como consequência, ela funciona melhor em \emph{datasets} nos quais o emprego de palavras varia significativamente por perspectiva. A fim de investigar como o uso de palavras, refletido em suas contagens, se associa ao desempenho da classificação, o Capítulo \ref{chap3} apresenta alguns experimentos neste sentido. Dois corpora são classificados com um Naïve Bayes baseado em contagens de palavras. Via validação cruzada, as taxas de acerto obtidas foram, respectivamente, de 86.22\% e 54.01\%. Isso significa que, no primeiro corpus, as contagens de palavras foram suficientes para identificar corretamente a classe dos documentos de forma satisfatória; no segundo corpus, tem-se o caso contrário. Em outras palavras, isso significa que, no primeiro corpus, o emprego de palavras varia significativamente por perspectiva; no segundo, não. 

A fim de verificar quais palavras foram mais enfocadas por cada ponto de vista, ampliando a compreensão dos resultados obtidos com o Naïve Bayes, foram feitos experimentos com o modelo L-LDA\footnote{O L-LDA é apresentado no Capítulo \ref{basicos}.}. O modelo associa palavras a tópicos, facilitando a visualização de quais são mais relevantes para cada um deles. Nesses experimentos, cada perspectiva de cada \emph{dataset} corresponde a um tópico, e um último tópico, neutro, é associado a todos os documentos. A finalidade do tópico neutro é filtrar palavras muito comuns em cada corpus, independentemente de perspectiva. Com isso, evidencia-se quais palavras receberam mais destaque por cada perspectiva. Selecionando-se as dez palavras mais frequentemente associadas a cada perspectiva, já é possível visualizar que, no primeiro corpus, perspectivas diferentes enfatizam ideias mais distintas do que no segundo corpus. Isso amplia a compreensão das classificações feitas com o Naïve Bayes, apresentando cenários em que o uso exclusivo de contagens de palavras é ou não suficiente para uma boa classificação. É importante frisar que não se encontrou nenhum outro trabalho que faça uso de um L-LDA para compreender, ainda que parcialmente, como certos termos são enfocados por diferentes perspectivas.

Aproveitando as revisões e discussões exploradas no Capítulo \ref{chap3}, decidiu-se fazer um estudo de caso envolvendo um \emph{dataset} brasileiro. Não foi encontrado nenhum outro trabalho que aplique as técnicas dessa sub-área de Mineração de Opinião a documentos escritos em português - em particular, envolvendo conteúdo brasileiro. Considerando que 2010 é ano de eleições presidenciais no Brasil, a abundância de artigos que carregam pontos de vista típicos da oposição e da situação foi explorada. Foi construído um corpus sobre o atual governo e as eleições, composto de material coletado em colunas, \emph{sites} e \emph{blogs} mantidos por jornalistas de notoriedade nacional. Em seguida, esse corpus foi dividido entre as perspectivas pró-situação e pró-oposição, e foi classificado com um Naïve Bayes. As características dos documentos exploradas por esse classificador foram suas contagens de palavras. Os resultados obtidos com essa metodologia foram muito positivos: a taxa de acerto, em particular, foi de 89.43\%. Isso significa que esses jornalistas consolidam suas perspectivas sobre a política brasileira já nas palavras que destacam. De fato, alguns trechos elencados no Capítulo \ref{estudo}, no qual o estudo é apresentado, sugerem que seus pontos de vista são defendidos com muita veemência. 

Esse estudo de caso também investiga, assim como no Capítulo \ref{chap3}, quais palavras receberam mais destaque por cada perspectiva. A análise com um L-LDA, aplicado de forma semelhante àquela do Capítulo \ref{chap3}, indica que os jornalistas pró-situação dão mais ênfase ao candidato de oposição José Serra do que os pró-oposição. Esses últimos, por sua vez, dão maior destaque ao presidente Lula e à sua candidata, Dilma Rousseff - ambos correspondem à situação. Isso sugere que os jornalistas enfatizam o ataque aos candidatos que se opõem, ideologicamente, aos pontos de vista que eles defendem. Isso reforça a ideia de que a defesa de um posicionamento, muitas vezes, compreende o ataque a posicionamentos opostos, como sugerem Somasundaran e Wiebe em seu trabalho sobre debates \emph{online} \cite{wiebe}.  

\section{Dificuldades encontradas}

A classificação de documentos de acordo com seus pontos de vista sobre um tema é um problema relativamente novo. De fato, ele só foi estabelecido como sub-área da Mineração de Opinião em 2008, na \emph{survey} de Pang e Lee. Por este motivo, para entender melhor o problema, foi necessário buscar artigos em diversas conferências que envolviam a área de Mineração de Opinião. A filtragem de quais resultados realmente tinham a ver com o tema dessa monografia não foi exatamente difícil, mas envolveu um trabalho manual considerável. Para o estudo de caso, a extração, filtragem e padronização dos documentos que compõem o corpus também envolveu algum trabalho manual. Embora essas tarefas tenham sido realizadas de forma automatizada, a construção dos \emph{scripts} dependia da compreensão de como o conteúdo estava disposto em cada veículo.  

\section{Trabalhos futuros}

Futuramente, pretende-se estender o estudo de caso do Capítulo \ref{estudo} a textos políticos escritos por cidadãos comuns em seus \emph{blogs}, o que pode contribuir positivamente para a compreensão de como o brasileiro se posiciona politicamente na Web. Além disso, o estudo também deve ser ampliado para identificar as perspectivas contidas nos comentários feitos aos artigos do corpus, em seus \emph{blogs}, \emph{sites} e colunas, a fim de se avaliar como eles refletem o posicionamento dos leitores em relação àquilo que leram. Este tipo de análise pode ajudar a compreender o impacto destes artigos em seus leitores e a formação de posicionamentos na mídia brasileira \emph{online}. Caso essas tarefas de classificação não sejam bem resolvidas com o uso exclusivo de contagens de palavras, esforços direcionados na busca de características mais complexas, envolvendo aspectos sintáticos/semânticos dos documentos, devem ser feitos. É válido ressaltar que isso provavelmente implicaria no desenvolvimento de ferramentas gramaticais de apoio, voltadas para a língua portuguesa.


