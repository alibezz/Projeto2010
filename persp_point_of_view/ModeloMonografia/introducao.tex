\chapter{Introdução}

%\section{Motivação}

%; antes de comprar um produto, muitas vezes procura-se relatos sobre a satisfação de outros consumidores.

Antes de assistir a um filme ou comprar um certo produto, é comum ler críticas a seu respeito ou considerar os comentários de outras pessoas. De forma semelhante, para se formular um posicionamento sobre algum tema - como aborto ou pena de morte, por exemplo - é comum levar em conta os pontos de vista de outros indivíduos. Com a disseminação da Web, a busca por textos opinativos ou argumentativos sofreu um aumento expressivo, passando a fazer parte do cotidiano de seus usuários. Naturalmente, isso ampliou os insumos para a formação de opinião desses usuários, causando um impacto tanto no processo de consumo quanto na defesa de pontos de vista a respeito dos mais diversos assuntos. Uma pesquisa realizada nos Estados Unidos reforça essa tese \cite{pesquisa-eua}, revelando que as opiniões expressas em resenhas \emph{online}, como críticas de restaurantes e albergues, influenciou fortemente as decisões de 73 a 87\% dos consumidores.

Diante disso, estudos com o intuito de extrair opiniões e pontos de vista da Web, e interpretá-los automaticamente, tornaram-se mais frequentes na área de Ciência da Computação. Juntos, esses estudos compõem o que ficou conhecido como \textbf{Análise de Sentimento} ou \textbf{Mineração de Opinião} \cite{omsa,bingliu}. De acordo com a \emph{survey} de Pang e Lee, a Mineração de Opinião é o emprego de diversas técnicas computacionais com o intuito de explorar algum dos tópicos abaixo \cite{omsa}:
% \cite{bingliu}.


% Segundo uma pesquisa realizada nos E(CITAÇÃO), a opinião expressa em resenhas online
%influenciou fortemente a decisão de 73 a 87% dos consumidores. Não?
   

%sofreu um aumento expressivo, passando a fazer parte do cotidiano dos 

% opiniões e pontos de vista sobre os mais diversos assuntos ganhou uma nova escalaa busca por opiniões, argumentos e pontos de vista em  

%A busca por opiniões sempre desempenhou um papel importante na geração de novas escolhas.

% Com a disseminação da Web e da Internet, a geração de opiniões com impacto, sobre os mais diversos assuntos, foi finalmente democratizada: não é mais preciso, por exemplo, ser um especialista em Economia ou Ciência Política para manter um \emph{blog} convincente sobre algum candidato às eleições.

%Neste contexto, a busca por opiniões e comentários em \emph{sites}, \emph{blogs}, fóruns e redes sociais também se popularizou, passando a fazer parte do cotidiano dos consumidores \emph{online}. Uma pesquisa feita nos Estados Unidos revelou que entre 73\% e 87\% dos leitores de resenhas de serviços \emph{online}, como críticas de restaurantes e albergues, sentem-se fortemente influenciados a consumi-los ou não a depender das opiniões contidas nessas resenhas \cite{pesquisa-eua}. Diante da relevância que opiniões têm na geração de decisões e no processo de consumo, estudos com o intuito de extraí-las da Web e interpretá-las automaticamente tornaram-se mais frequentes na área de Ciência da Computação. Juntos, esses estudos compõem o que ficou conhecido como \textbf{Análise de Sentimento} ou \textbf{Mineração de Opinião}, termos utilizados como sinônimos \cite{omsa} \cite{bingliu}.


\begin{enumerate}
    \item \textbf{Polaridade de sentimento ou graus de polaridade -} Dado um documento opinativo, para o qual se assume que as opiniões se referem a um único assunto, classifique-o como expressando uma opinião estritamente positiva, estritamente negativa, ou em algum grau bem definido entre esses dois extremos;
%classifique-o como positivo ou negativo em relação a esse assunto (polaridades opostas) ou localize-o no espectro estabelecido entre duas polaridades opostas
    \item \textbf{Detecção de subjetividade e identificação de opinião -} Dado um documento, detecte se ele é subjetivo ou não, i.e. se consiste de opiniões ou de fatos, ou que partes dele são subjetivas;
    \item \textbf{Análise de tópico-sentimento -} Dado um documento opinativo, assume-se que suas opiniões podem se referir a tópicos diferentes, e deve-se identificar quais opiniões são sobre quais tópicos;
   \item \textbf{Pontos de vista ou perspectivas -} Dado um documento opinativo, que apresente um ponto de vista\footnote{Os termos \emph{posicionamento}, \emph{orientação}, \emph{perspectiva}, \emph{ponto de vista} e \emph{ideologia} são utilizados de forma intercambiável nessa monografia, por serem explorados da mesma forma na literatura revisada para este projeto.} sobre um tema, em vez de um sentimento polarizado sobre um único assunto, classifique-o de acordo com esse ponto de vista;
   \item \textbf{Outras informações não-factuais -} Dado um documento com caráter emotivo/sentimental, identifique que tipos de humor o permeiam e/ou classifique-o de acordo com as emoções encontradas.

\end{enumerate}

O item (1) recebe mais destaque, funcionando como base para ferramentas de monitoramento de conteúdo \emph{online}\footnote{O TwitterSentiment (http://twittersentiment.appspot.com/) e o Moodviews (http://moodviews.com/) são dois exemplos desse tipo de ferramenta.}. Essas ferramentas se popularizaram na área de \emph{marketing}, sendo utilizadas para avaliar como pessoas públicas, produtos e marcas são comentados na Web. Basicamente, elas fixam um produto, marca ou pessoa pública e classificam as opiniões sobre ele como positivas ou negativas. Dentre os outros itens, o (4) tem recebido  atenção crescente como área de pesquisa - em 2010, inclusive, uma conferência importante para a área de Mineração de Opinião, a \emph{AAAI  Conference}\footnote{\emph{Association for the Advancement of Artificial Intelligence Conference}.}, lançou uma seção específica para trabalhos nesta direção: \emph{Sentiment and Perspective}\footnote{O programa dessa conferência, que contém essa seção, pode ser conferido em www.aaai.org/Conferences/AAAI/2010/aaai10schedule.pdf}. 

Uma diferença fundamental entre os dois itens diz respeito à natureza das classes: enquanto, no item (1), elas correspondem normalmente aos pólos positivo ou negativo, no item (4) elas correspondem a diferentes pontos de vista associados a um mesmo tema. Neste último caso, a ideia é classificar os documentos de acordo com esses pontos de vista, que se associam às crenças e atitudes de seus autores em relação a um conjunto de múltiplos assuntos (uma temática) \cite{omsa}. Não se trata, portanto, de classificar conteúdo como estritamente positivo ou negativo, mas sim como alinhado à defesa de um determinado posicionamento, como \emph{pró-aborto} ou \emph{neoliberal}. 

Embora a diferença entre opiniões e pontos de vista seja um tanto subjetiva, no contexto da Mineração de Opinião as primeiras têm sido associadas a eventos pontuais, envolvendo marcas/produtos/pessoas públicas; os segundos, por sua vez, são relacionados a temáticas de cunho social, envolvendo argumentações que revelam ideologias e crenças. O item (4), portanto, pode ser aplicado à compreensão de como as pessoas, na Web, argumentam em defesa dessas ideologias e crenças. Neste sentido, esse item ataca computacionalmente um problema que já era estudado em outras áreas, como Comunicação e Ciências Políticas \cite{gentzkow, milyo, fader}: a investigação dos pontos de vista contidos em textos. Diante disso, e da crescente atenção que este item vem recebendo, ele foi escolhido como o enfoque dessa monografia. 

Essa monografia apresenta uma revisão bibliográfica da área de classificação por ponto de vista, abordada superficialmente na \emph{survey} de Pang e Lee, acompanhada de um estudo de caso. Inicialmente, é feita uma descrição básica dos principais classificadores utilizados na área, no Capítulo \ref{basicos}. Eles são o Naïve Bayes e os \emph{Support vector machines} (SVMs), muito populares na área de Aprendizado de Máquina. Neste capítulo também são apresentadas métricas para se avaliar o desempenho de uma classificação e uma técnica que valida essa avaliação. Ainda nesse capítulo, por fim, é apresentado o modelo \emph{Labeled-latent dirichlet allocation} (L-LDA), que associa documentos a tópicos e relaciona suas palavras a cada um deles. No Capítulo \ref{chap3}, trabalhos sobre classificação por ponto de vista são selecionados, revisados e analisados de forma comparativa. No Capítulo \ref{estudo}, é apresentado um estudo de caso envolvendo a política brasileira. O escopo desse capítulo envolve a construção de um corpus\footnote{Nesta monografia, os termos \emph{corpus} e \emph{dataset} serão utilizados de forma intercambiável, e se referem a um conjunto de documentos de texto.}, a definição dos pontos de vista a serem considerados, a classificação de documentos de acordo com eles e a discussão do uso de palavras por cada posicionamento. Por fim, no Capítulo \ref{conclusoes}, são apresentadas conclusões a respeito dos conteúdos explorados pelos Capítulos \ref{chap3} e \ref{estudo}. Esse capítulo também discute as principais dificuldades encontradas nessa monografia e trabalhos futuros. As próximas seções apresentam, respectivamente, os objetivos dessa monografia e alguns trabalhos relacionados.



\section{Objetivo}
\label{objetivo}

O objetivo desse trabalho é explorar aspectos teóricos e práticos da classificação de documentos de acordo com seus pontos de vista. Para a parte teórica, foi elaborada uma revisão de artigos que tratam do assunto. Para a parte prática, foi proposto um estudo de caso envolvendo um corpus de política brasileira, explorando todos os passos envolvidos na sua classificação e uma análise de como as palavras são enfatizadas por cada ponto de vista. 

%A ideia é disponibilizar um documento que contibua para estudos futuros na área, considerando aspectos teóricos e práticos. 
%; além disso, a fim de aprofundar a compreensão dos trabalhos revisados, discute-se a relação entre o uso de palavras e a classificação, com o apoio de alguns experimentos
\section{Trabalhos Relacionados}
\label{relacionados}

Classificação de documentos por ponto de vista é uma área recente, que vem se popularizando nos principais eventos de Processamento de Linguagem Natural. Apesar disso, ainda há pouco material de apoio para uma introdução ao assunto, sendo necessário ir diretamente aos \emph{sites} desses eventos em busca de artigos. O principal trabalho encontrado nessa direção foi justamente a \emph{survey} de Pang e Lee, que propõe a temática como um subproblema da área de Mineração de Opinião \cite{omsa}. Esse trabalho apresenta uma boa introdução à área de Mineração de Opinião, apresentando suas principais aplicações, desafios, técnicas e métodos diretamente envolvidos com a identificação, extração e análise de informação opinativa. Dada a abrangência da \emph{survey}, a pouca atenção dedicada à temática dessa monografia era esperada. De todo modo, esse material é aquele que mais se aproxima dos objetivos desse projeto.

No que diz respeito à Mineração de Opinião em geral, Bing Liu propõe materiais introdutórios desde 2006 \cite{bingliu} \cite{handbook-liu}. O enfoque desses trabalhos é apresentar as principais definições da área, seus problemas e técnicas para resolvê-los. Apesar de se aproximar, em diversos pontos, da \emph{survey} de Pang e Lee, seus trabalhos não possuem um caráter de revisão, nem apresentam uma divisão clara da área em subproblemas. Por estes motivos, os trabalhos de Bing Liu funcionam como um bom material de apoio para essa monografia, mas não se relacionam tão intimamente com seus objetivos como a \emph{survey} de Pang e Lee.

Os trabalhos relacionados ao estudo de caso desse projeto, explorado no Capítulo \ref{estudo}, consistem na revisão em si, apresentada no Capítulo \ref{chap3}. São diversos artigos que se propõem a classificar documentos de acordo com seus pontos de vista, utilizando uma mesma metodologia básica. No que diz respeito à temática explorada, a política brasileira, a ferramenta Eleitorando\footnote{http://www.eleitorando.com.br/site/} se aproxima deste projeto. A sua finalidade, entretanto, é monitorar opiniões sobre candidatos às Eleições 2010 nas redes sociais, como Twitter\footnote{http://twitter.com/} ou YouTube\footnote{http://br.youtube.com/}. As informações monitoradas são classificadas como \emph{positivas}, \emph{negativas} ou \emph{neutras}. Não se trata, portanto, do enfoque proposto pela classificação por ponto de vista, que busca identificar o posicionamento defendido em um documento, em vez de opiniões polarizadas. Por fim, o uso de um L-LDA para aprofundar a compreensão dos resultados obtidos com a classificação não foi encontrado em nenhum trabalho estudado para esse projeto.     
%revisão: bing liu / blabla
% parte prática: ver o chap3



%um modelo utilizado no Capítulo \ref{estudo}, com a finalidade de ilustrar o uso de palavras por documentos escritos sob perspectivas diferentes. O modelo é  Não foi encontrado \textbf{nenhum} outro trabalho que utilize esse modelo da forma apresentada no Capítulo \ref{estudo}. 

%A aplicação deles compõe a metodologia básica de quase todos os trabalhos estudados - o que varia, de fato, são as representações dos documentos e a forma como eles são pré-processados. 

%Como a \emph{survey} de Pang e Lee elenca apenas três trabalhos que envolvem classificação de documentos por ponto de vista, um dos objetivos principais dessa monografia foi explorar essa revisão, criando um documento que possa contribuir com estudos futuros nessa linha. Os trabalhos revisados foram escolhidos de acordo com metodologia definida no Capítulo \ref{chap3}, e são apresentados no mesmo. Grande parte deles classifica documentos baseando-se em como eles usam palavras. A ideia, que remete a estudos de Linguística, é de que indivíduos com posicionamentos diferentes utilizam as palavras de formas distintas  \cite{teubert} - e isso é explorado na classificação. 



%Além da revisão de trabalhos, esse projeto propõe um estudo de caso envolvendo posicionamentos sobre a política brasileira no Capítulo \ref{estudo}. Além de propor uma classificação por ponto de vista, esse capítulo também discute o uso de palavras por cada uma delas. Não foi encontrado \textbf{nenhum} trabalho que classifique documentos brasileiros de acordo com seus pontos de vista, o que faz desse estudo, pelo que tudo indica, o primeiro envolvendo uma temática brasileira. Os resultados são animadores, indicando que a classificação de documentos de acordo com seus pontos de vista também pode ser aplicada, de forma bem sucedida, a \emph{datasets}\footnote{\emph{Datasets} são conjuntos de dados; nesse caso, de documentos.} em português.



%Este capítulo também discute a relação entre o uso de palavras nos documentos e o desempenho da classificação.

% de documentos escritos em \textbf{português}  revisar artigos que classificam documentos de acordo com suas perspectivas no Capítulo \ref{chap3}, 

%A \emph{survey} de Pang e Lee elenca seis trabalhos relacionados ao item \textbf{Pontos de vista ou perspectivas}, e apenas três deles têm diretamente a ver com classificação de documentos. Os outros envolvem   

%É válido ressaltar que não foi encontrado nenhum trabalho especificamente dedicado à revisão de estudos sobre \textbf{Pontos de vista ou perspectivas}.

%o que o seu trabalho  borda é exatamente uma tentativa de pegar esses problemas que as pessoas já estudavam em outras áreas e atacar computacionalmente. 


% baseadas em crenças, ideologias  de crenças, ideologias e argumentações  os documentos podem ser classificados em mais de dois pontos de vista. Embora a diferença entre opinião e ponto de vista seja um tanto subjetiva, ela é adotada , portanto, por se mostrar útil no contexto de classificação. , que não necessariamente se. Além disso, os pesquisadores  devem ser classificados como  apresentando grande aplicação 

%em particular, é de grande aplicabilidade para as Ciências Humanas. Diferentemente dos outros tópicos, mais aplicáveis à compreensão de opiniões \textbf{pontuais} sobre marcas/produtos/pessoas públicas, esse item investiga um fenômeno mais profundo: o posicionamento\footnote{Os termos \emph{posicionamento}, \emph{orientação}, \emph{perspectiva}, \emph{ponto de vista} e \emph{ideologia} são utilizados de forma intercambiável nessa monografia, por serem explorados da mesma forma na literatura revisada para este projeto.} de indivíduos a respeito de temas mais \textbf{abrangentes}. Não se trata, portanto, de investigar opiniões estritamente polarizadas, como \emph{positivo}, \emph{negativo}, \emph{bom} ou \emph{ruim} - mas sim pontos de vista e ideologias, como \emph{pró-aborto} ou \emph{anti-pena de morte}. Estas questões motivaram o enfoque dessa monografia para o item. Em outras palavras, a temática explorada por essa monografia é a \textbf{classificação de documentos de acordo com seus pontos de vista sobre um tema}.

%Embora a diferenciação entre opinião e ponto de vista seja um tanto subjetiva, ela é adotada por se mostrar útil no contexto de classificação. Em trabalhos que classificam documentos de acordo com suas opiniões, fixa-se um objeto e se separam duas classes: uma corresponde a menções positivas; outra, a menções negativas \cite{omsa}. Quando o foco é em pontos de vista, a ideia é separar os documentos de acordo com eles, que correspondem às crenças e atitudes de seus autores em relação a um conjunto de múltiplos assuntos (uma temática) \cite{omsa}. Neste sentido, os documentos podem ser separados, em tese, em mais de dois pontos de vista.

%É válido ressaltar que a investigação dos pontos de vista contidos em documentos também é explorada em trabalhos de outras áreas, como Comunicação e  Ciências Políticas \cite{gentzkow} \cite{milyo} \cite{fader}. Isso evidencia a interdisciplinaridade da temática dessa monografia. 

                              


%Considerando diversos artigos lidos para esse projeto, foi observado também que documentos classificados por opiniões costumam conter muitos adjetivos polarizados (como \emph{bom} ou \emph{ruim}), que, em alguns casos, são explorados na classificação de forma diferenciada \cite{thumbs-up} \cite{peanut-gallery}. Isso não foi observado em trabalhos que fazem classificação por ponto de vista. Embora, na prática, a classificação em ambos os casos possa ser executada utilizando técnicas semelhantes, essas diferenças podem implicar em pré-processamentos dos documentos bem distintos, ou ainda na exploração de características diferentes dos \emph{datasets}.  , que não necessariamente fazem um uso destacado de adjetivos.  quais ideias receberam mais destaque, ou quais ângulos de um mesmo tema são explorados por posicionamentos distintos.   Quando o foco é em pontos de vista, 
 


%    fale de interdisc.                                                                                                       Note
%that a number of other aspects of politically-oriented text, such as whether liberal or conservative views are
%expressed, have been explored; since the labels used in those problems can usually be considered properties
%of a set of documents representing authors’ attitudes over multiple issues rather than positive or negative sen-
%timent with respect to a single issue,



%A revisão executada nesse projeto levou a discussões e experimentos também apresentados no Capítulo \ref{chap3}, que visam à investigação de como o uso das palavras interfere na classificação de documentos por ponto de vista. Esses experimentos ilustram aspectos interessantes dos documentos, ampliando a discussão da classificação por ponto de vista. Não foi encontrado \textbf{nenhum} outro trabalho que apresente experimentos semelhantes aos apresentados no Capítulo \ref{chap3} - ou seja, além da revisão em si, tudo indica que a execução desses experimentos é uma contribuição inédita para a área de Mineração de Opinião. 
   


%one might be seeking something more like “achieving world peace is difficult” than like “mildly positive”. In fact, much of the prior work on perspectives and viewpoints seeks to extract more perspective-related information (e.g., opinion holders). 



% simples como \emph{bom} ou \emph{ruim}, mas sim pontos de vista razoavelmente complexos, como "eu acho que a paz mundial

%Tais posicionamentos não necessariamente enfocam palavras polarizadas, como \emph{bom} ou \emph{ruim}. 

%a classificação de documentos de acordo com seus pontos de vista colabora com uma compreensão mais profunda de como determinados grupos sociais se posicionam a respeito de determinados temas. Em vez de opiniões pontuais, ode colaborar na análise do impacto sociológico de determinados assuntos. 


% Entender o posicionamento de um determinado público, através de seus documentos, a respeito de um determinado tema pode encadear uma série de estudos sobre o impacto desse tema neste grupo social, suas causas, consequências e demais relações. % Os artigos revisados na \emph{survey} de Pang e Lee, bem como outros pesquisados para este projeto, enfocam na compreensão de

%No começo dos anos 2000, os termos \emph{Opinion Mining} e \emph{Sentiment Analysis} são empregados pela primeira vez, referindo-se ao emprego de técnicas computacionais para identificação e avaliação de opiniões e sentimentos em textos \cite{panglee}. Esses termos têm funcionado como sinônimos \cite{bing}, aparecendo paralelamente na literatura. De acordo com \cite{panglee}, o termo \emph{Sentiment Analysis} é empregado pela primeira vez em 2001, nos artigos \textbf{Como citar referências citadas em outro trabalho?}; o termo \emph{Opinion Mining}, em 2003, no artigo \textbf{mesma duvida}. Nesta monografia, os dois termos serão utilizados de forma intercambiável.


%A ideia de utilizar metodologias computacionais para identificar e analisar opiniões é muito anterior à popularização da Web \textbf{Citar artigos do fim da década de 60 e começo de 70 que provam isso}. Motivos: pouco dado, IR e ML imaturas. Explicar os 3 e como se relacionam com Natural Language Processing.
%\textbf{POPULARIDADE DAS ÁREAS DE ML E IR, SURGIMENTO DE DATASETS DISPONIVEIS POR CONTA DA POPULARIDADE DA INTERNET...}.

%One of the main reasons for the lack of study on opinions is that there was little opinionated text before
%the World Wide Web. Before the Web, when an individual needs to make a decision, he/she typically
%asks for opinions from friends and families. 

%Falar das principais tasks da área.
