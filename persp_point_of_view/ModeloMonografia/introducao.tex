\chapter{Introdução}

\section{Motivação}

A busca por opiniões sempre desempenhou um papel importante na geração de novas escolhas. Antes de optar por assistir a um filme, é comum ler críticas a seu respeito ou considerar os comentários de outras pessoas; antes de comprar um produto, muitas vezes procuramos relatos sobre a satisfação de outros consumidores. Com a disseminação da Web e da Internet, a geração de opiniões com impacto, sobre os mais diversos assuntos, foi finalmente democratizada: não é mais preciso, por exemplo, ser um especialista em Economia ou Ciência Política para manter um blog \textbf{deveria definir blog?} convincente sobre algum candidato às eleições.

Neste contexto, a busca por opiniões e comentários em sites, blogs, fóruns e redes sociais também se popularizou, passando a fazer parte do cotidiano dos consumidores online. Uma pesquisa feita nos Estados Unidos revela que entre 73\% e 87\% dos leitores de resenhas de serviços online, como críticas de restaurantes e albergues, sentem-se fortemente influenciados a consumi-los ou não a depender das opiniões contidas nessas resenhas \cite{pesquisa-eua}. Diante da relevância que opiniões têm na geração de decisões e no processo de consumo, estudos com o intuito de extraí-las da Web e interpretá-las automaticamente tornaram-se mais frequentes na área de Ciência da Computação. Juntos, esses estudos compõem o que ficou conhecido como \textbf{Análise de Sentimento} ou \textbf{Mineração de Opinião}\footnote{Os dois termos, por serem considerados sinônimos, serão utilizados de forma intercambiável no decorrer desta monografia}.

De acordo com \cite{pang-lee-survey}, a área envolve o emprego de diversas técnicas computacionais com o intuito de atingir algum - ou alguns - dos objetivos abaixo:

\begin{enumerate}
   \item \textbf{Identificação de opinião –} Dado um conjunto de documentos, separe fatos de opiniões;
   \item \textbf{Avaliação de polaridade -} Dado um conjunto de documentos com caráter opinativo e uma palavra-chave (figura pública, empresa etc), classifique as opiniões como positivas ou negativas, ou indique o grau de negatividade/positividade de cada uma delas;
   \item \textbf{Classificação de pontos de vista ou perspectivas -} Dado um conjunto de documentos contendo perspectivas ou pontos de vista sobre um mesmo tema/conjunto de temas, classifique-os de acordo com essas perspectivas/pontos de vista;
   \item \textbf{Reconhecimento de humor -} Dado um conjunto de textos com caráter emotivo/sentimental, como posts de blogs pessoais, identifique que tipos de humor permeiam os textos e/ou classifique-os de acordo com as diferentes emoções encontradas.

\end{enumerate}

%No começo dos anos 2000, os termos \emph{Opinion Mining} e \emph{Sentiment Analysis} são empregados pela primeira vez, referindo-se ao emprego de técnicas computacionais para identificação e avaliação de opiniões e sentimentos em textos \cite{panglee}. Esses termos têm funcionado como sinônimos \cite{bing}, aparecendo paralelamente na literatura. De acordo com \cite{panglee}, o termo \emph{Sentiment Analysis} é empregado pela primeira vez em 2001, nos artigos \textbf{Como citar referências citadas em outro trabalho?}; o termo \emph{Opinion Mining}, em 2003, no artigo \textbf{mesma duvida}. Nesta monografia, os dois termos serão utilizados de forma intercambiável.


A ideia de utilizar metodologias computacionais para identificar e analisar opiniões é muito anterior à popularização da Web \textbf{Citar artigos do fim da década de 60 e começo de 70 que provam isso}. Motivos: pouco dado, IR e ML imaturas. Explicar os 3 e como se relacionam com Natural Language Processing.
%\textbf{POPULARIDADE DAS ÁREAS DE ML E IR, SURGIMENTO DE DATASETS DISPONIVEIS POR CONTA DA POPULARIDADE DA INTERNET...}.

%One of the main reasons for the lack of study on opinions is that there was little opinionated text before
%the World Wide Web. Before the Web, when an individual needs to make a decision, he/she typically
%asks for opinions from friends and families. 

%Falar das principais tasks da área.

\section{Proposta}

Falar de Mineração de Perspectiva. Definir todos os termos correlatos utilizados, fechar os problemas da área e explicar como isso se diferencia de Opinion Mining clássica, que é basicamente Análise de Polaridade.

%\textbf{explicar que aqui muitas vezes é mais interessante entender posicionamentos globais do que a opinião sobre um único assunto, tópico, etc, e que tb há pesquisas na área de opinion mining que tratam disso; mostrar onde diferem, os problemas} Especificamente no que diz respeito à política, uma pesquisa realizada nos Estados Unidos \cite{rainiehorrigan}, com mais de 2500 usuários de Internet que compartilharam informação sobre as eleições de 2006 \emph{online}, concluiu o seguinte:

%\begin{enumerate}
 % \item Para 28\% deles, um dos principais motivos para realizar esta troca de informações \emph{online} era compreender as perspectivas de suas próprias comunidades; 
  %\item 34\% deles afirmaram que um dos principais motivos para realizar esta troca de informações \emph{online} era compreender as perspectivas exteriores à suas comunidades; 
   %\item 27\% deles pesquisaram avaliações políticas de organizações externas \emph{online};
%  \item 28\% deles afirmaram que a maioria dos \emph{sites} de política que frequentam compartilham com seus pontos de vista;
 % \item 29\% deles afirmaram que a maioria dos \emph{sites} de política que frequentam apresentam pontos de vista que divergem dos seus - indicação de que muitas pessoas não estão buscando apenas uma validação de opiniões pré-existentes\cite{panglee};
  %\item 8\% deles publicam seus próprios comentários políticos \emph{online}.
%\end{enumerate}  

%\textbf{Concluir fechando mais o escopo para perspectivas; tentar motivar um pouco para eleições, MAAAAAS falar brevemente de trabalhos que não necessariamente tem a ver com política e que tb tentam capturar perspectivas/pontos de vista}.

%Falar que as bases abrangentes serão discutidas, mas essa mono fecha o escopo em viewpoints, blablabla. Falar que se pretende fazer uma revisão comparativa dos trabalhos, mostrar alguma coisa prática na área com as decisões de projeto devidamente justificadas.

\section{Estrutura da Monografia}

Falar da metodologia de busca dos artigos 
%Modelo de monografia usando as normas ABNT (Associação Brasileira de Normas
%%Técnicas) \sigla{ABNT}{Associação Brasileira de Normas Técnicas}
%e adaptação personalizada 
%do padrão do Departamento de Ciência da Computação (DCC) da Universidade
%Federal da Bahia (UFBA).
%\sigla{DCC}{Departamento de Ciência da Computação}
%\sigla{UFBA}{Universidade Federal da Bahia}
%Fontes latex cedidos pela ABNT e disponibilizados por 
%Maurício Vieira. (Valeu Maurix!). Adaptado por Abelmon Bastos por solicitação
%da \profa\ Débora Abdalla para o semestre 2005.1.
%Adaptado por Rodrigo Rocha por solicitação da \profa\ Débora Abdalla no fim
%do semestre 2007.1.

% use sua propria estrutura

%Problema etc

%Objetivo etc

%Resultados esperados etc

%Estrutura/Organização etc
