\chapter{Introdução}

%\section{Motivação}

A busca por opiniões sempre desempenhou um papel importante na geração de novas escolhas. Antes de optar por assistir a um filme, é comum ler críticas a seu respeito ou considerar os comentários de outras pessoas; antes de comprar um produto, muitas vezes procuramos relatos sobre a satisfação de outros consumidores. Com a disseminação da Web e da Internet, a geração de opiniões com impacto, sobre os mais diversos assuntos, foi finalmente democratizada: não é mais preciso, por exemplo, ser um especialista em Economia ou Ciência Política para manter um \emph{blog} convincente sobre algum candidato às eleições.

Neste contexto, a busca por opiniões e comentários em \emph{sites}, \emph{blogs}, fóruns e redes sociais também se popularizou, passando a fazer parte do cotidiano dos consumidores \emph{online}. Uma pesquisa feita nos Estados Unidos revelou que entre 73\% e 87\% dos leitores de resenhas de serviços online, como críticas de restaurantes e albergues, sentem-se fortemente influenciados a consumi-los ou não a depender das opiniões contidas nessas resenhas \cite{pesquisa-eua}. Diante da relevância que opiniões têm na geração de decisões e no processo de consumo, estudos com o intuito de extraí-las da Web e interpretá-las automaticamente tornaram-se mais frequentes na área de Ciência da Computação. Juntos, esses estudos compõem o que ficou conhecido como \textbf{Análise de Sentimento} ou \textbf{Mineração de Opinião}, termos utilizados como sinônimos \cite{omsa} \cite{bingliu}.

De acordo com a \emph{survey} de Pang e Lee, referência mais citada para o estudo da área, a Mineração de Opinião envolve o emprego de diversas técnicas computacionais com o intuito de explorar algum dos tópicos abaixo \cite{omsa}:

\begin{enumerate}
    \item \textbf{Polaridade de sentimento ou graus de polaridade -} Dado um documento opinativo, para o qual se assume que as opiniões se referem basicamente a um único assunto, classifique-o como positivo ou negativo em relação a esse assunto (polaridades opostas) ou localize-o no espectro estabelecido entre duas polaridades opostas;
    \item \textbf{Detecção de subjetividade e identificação de opinião -} Dado um documento, detecte se ele é subjetivo ou não, constituindo-se de fatos ou opiniões, ou que partes dele são subjetivas;
    \item \textbf{Análise de tópico-sentimento -} Dado um documento opinativo, assume-se que suas opiniões podem se referir a tópicos diferentes, e deve-se identificar quais opiniões interagem com quais tópicos;
   \item \textbf{Pontos de vista ou perspectivas -} Dado um documento opinativo, que apresente uma perspectiva sobre um tema (um ponto de vista, uma orientação ideológica, um posicionamento), em vez de um sentimento polarizado sobre um único assunto, classifique-o de acordo com essa perspectiva;
   \item \textbf{Outras informações não-factuais -} Dado um documento com caráter emotivo/sentimental, identifique que tipos de humor o permeiam e/ou classifique-o de acordo com as emoções encontradas.

\end{enumerate}

O item \textbf{Pontos de vista ou perspectivas}, em particular, é de grande aplicabilidade para as Ciências Humanas. Diferentemente dos outros tópicos, mais aplicáveis à compreensão de opiniões \textbf{pontuais} sobre marcas/produtos/pessoas públicas, esse item investiga um fenômeno mais profundo: o posicionamento\footnote{Os termos \emph{posicionamento}, \emph{orientação}, \emph{perspectiva}, \emph{ponto de vista} e \emph{ideologia} são utilizados de forma intercambiável nessa monografia, por serem explorados da mesma forma na literatura revisada para este projeto.} de indivíduos a respeito de temas mais \textbf{abrangentes}. Não se trata, portanto, de investigar opiniões estritamente polarizadas, como \emph{positivo}, \emph{negativo}, \emph{bom} ou \emph{ruim} - mas sim perspectivas e ideologias, como \emph{pró-aborto} ou \emph{anti-pena de morte}. Estas questões motivaram o enfoque dessa monografia para esse item. Em outras palavras, a temática explorada por essa monografia é a \textbf{classificação de documentos de acordo com suas perspectivas}.

Como a \emph{survey} de Pang e Lee elenca \textbf{apenas três} trabalhos que envolvem classificação de documentos por perspectiva, um dos objetivos principais dessa monografia foi explorar essa revisão, criando um documento que possa servir como referência para estudos futuros nessa linha. Os trabalhos revisados foram escolhidos de acordo com metodologia definida no Capítulo \ref{chap3}, e são apresentados no mesmo. Todos eles classificam documentos baseando-se em como eles usam palavras. A ideia, que remete a estudos de Linguística, é de que indivíduos com posicionamentos diferentes utilizam as palavras de formas distintas  \cite{teubert} - e isso é explorado na classificação. 

A revisão executada nesse projeto levou a discussões e experimentos também apresentados no Capítulo \ref{chap3}, que visam à investigação de como o uso das palavras interfere na classificação de documentos por perspectiva. Esses experimentos ilustram aspectos interessantes dos documentos, ampliando a discussão da classificação por perspectiva. Não foi encontrado \textbf{nenhum} outro trabalho que apresente experimentos semelhantes aos apresentados no Capítulo \ref{chap3} - ou seja, além da revisão em si, tudo indica que a execução desses experimentos é uma contribuição inédita para a área de Mineração de Opinião. 

Além da revisão acompanhada de experimentos, esse projeto propõe um estudo de caso envolvendo posicionamentos sobre a política brasileira no Capítulo \ref{estudo}. Além de propor uma classificação por perspectiva, esse capítulo também discute o uso de palavras por cada uma delas. Não foi encontrado \textbf{nenhum} trabalho que classifique documentos brasileiros de acordo com seus pontos de vista, o que faz desse estudo, pelo que tudo indica, o primeiro envolvendo uma temática brasileira. Os resultados são animadores, indicando que a classificação de documentos de acordo com seus pontos de vista também pode ser aplicada, de forma bem sucedida, a \emph{datasets}\footnote{\emph{Datasets} são conjuntos de dados; nesse caso, de documentos.} em português.


Nesse trabalho é feita, primeiramente, uma descrição básica dos principais classificadores explorados nessa monografia, no Capítulo \ref{basicos}. Eles são o Naïve Bayes e os \emph{Support Vector Machines} (SVMs), muito populares na área de Aprendizado de Máquina. A aplicação deles compõe a metodologia básica de todos os trabalhos estudados - o que varia, de fato, são as representações dos documentos e a forma como eles são pré-processados. No Capítulo \ref{basicos} também são apresentadas métricas para se avaliar o desempenho de uma classificação e uma técnica de validação para esse desempenho. Ainda nesse capítulo, por fim, é apresentado um modelo utilizado nos Capítulos \ref{chap3} e \ref{estudo}, com a finalidade de ilustrar o uso de palavras por documentos escritos sob perspectivas diferentes. O modelo é o \emph{Labeled-Latent Dirichlet Allocation} (L-LDA), que associa documentos a tópicos e relaciona suas palavras a cada um deles. No Capítulo \ref{chap3}, alguns trabalhos são selecionados e revisados de forma comparativa. Este capítulo também discute a relação entre o uso de palavras nos documentos e o desempenho da classificação. No Capítulo \ref{estudo}, é apresentado um estudo de caso envolvendo a política brasileira. O escopo desse capítulo envolve a construção de um corpus\footnote{Nesta monografia, os termos \emph{corpus} e \emph{dataset} serão utilizados de forma intercambiável.}, a definição dos pontos de vista a serem considerados, a classificação de documentos de acordo com eles, a discussão do uso de palavras e uma série de considerações finais e indicações de estudos futuros. Por fim, no Capítulo \ref{conclusoes}, são apresentadas - ou ratificadas - conclusões a respeito dos conteúdos explorados pelos Capítulos \ref{chap3} e \ref{estudo}. Esse capítulo também discute as principais dificuldades encontradas nesse projeto. As próximas seções apresentam, respectivamente, os objetivos desse trabalho e alguns trabalhos relacionados.


% de documentos escritos em \textbf{português}  revisar artigos que classificam documentos de acordo com suas perspectivas no Capítulo \ref{chap3}, 

%A \emph{survey} de Pang e Lee elenca seis trabalhos relacionados ao item \textbf{Pontos de vista ou perspectivas}, e apenas três deles têm diretamente a ver com classificação de documentos. Os outros envolvem   

%É válido ressaltar que não foi encontrado nenhum trabalho especificamente dedicado à revisão de estudos sobre \textbf{Pontos de vista ou perspectivas}.
   


%one might be seeking something more like “achieving world peace is difficult” than like “mildly positive”. In fact, much of the prior work on perspectives and viewpoints seeks to extract more perspective-related information (e.g., opinion holders). 



% simples como \emph{bom} ou \emph{ruim}, mas sim pontos de vista razoavelmente complexos, como "eu acho que a paz mundial

%Tais posicionamentos não necessariamente enfocam palavras polarizadas, como \emph{bom} ou \emph{ruim}. 

%a classificação de documentos de acordo com seus pontos de vista colabora com uma compreensão mais profunda de como determinados grupos sociais se posicionam a respeito de determinados temas. Em vez de opiniões pontuais, ode colaborar na análise do impacto sociológico de determinados assuntos. 


% Entender o posicionamento de um determinado público, através de seus documentos, a respeito de um determinado tema pode encadear uma série de estudos sobre o impacto desse tema neste grupo social, suas causas, consequências e demais relações. % Os artigos revisados na \emph{survey} de Pang e Lee, bem como outros pesquisados para este projeto, enfocam na compreensão de

%No começo dos anos 2000, os termos \emph{Opinion Mining} e \emph{Sentiment Analysis} são empregados pela primeira vez, referindo-se ao emprego de técnicas computacionais para identificação e avaliação de opiniões e sentimentos em textos \cite{panglee}. Esses termos têm funcionado como sinônimos \cite{bing}, aparecendo paralelamente na literatura. De acordo com \cite{panglee}, o termo \emph{Sentiment Analysis} é empregado pela primeira vez em 2001, nos artigos \textbf{Como citar referências citadas em outro trabalho?}; o termo \emph{Opinion Mining}, em 2003, no artigo \textbf{mesma duvida}. Nesta monografia, os dois termos serão utilizados de forma intercambiável.


%A ideia de utilizar metodologias computacionais para identificar e analisar opiniões é muito anterior à popularização da Web \textbf{Citar artigos do fim da década de 60 e começo de 70 que provam isso}. Motivos: pouco dado, IR e ML imaturas. Explicar os 3 e como se relacionam com Natural Language Processing.
%\textbf{POPULARIDADE DAS ÁREAS DE ML E IR, SURGIMENTO DE DATASETS DISPONIVEIS POR CONTA DA POPULARIDADE DA INTERNET...}.

%One of the main reasons for the lack of study on opinions is that there was little opinionated text before
%the World Wide Web. Before the Web, when an individual needs to make a decision, he/she typically
%asks for opinions from friends and families. 

%Falar das principais tasks da área.

\section{Objetivo}
\label{objetivo}

Neste sentido, ele provê uma revisão de artigos que tratam o assunto; discute a relação  fomentar estudos, através de uma revisão acompanhada de discussão e um estudo de caso que mostra, passo a passo, como classificar de forma bem básica, mas conseguindo um desempenho legal, e como isso pode ter aplicações hiper legais.

%    Nesse trabalho pretendemos estudar e modelar computacionalmente o processo de análise harmônica funcional de música tonal, quantificando-o o máximo possível, e comparando solu-ções propostas e preexistentes com critérios objetivos. Para fazer isso, estudamos como pode ser aplicada a técnica de modelos de Markov ocultos nesse problema e avaliamos sua perfor-mance comparando-a com a do algoritmo de Pardo e Birmingham (PARDO; BIRMINGHAM,1999), o algoritmo de Tsui (TSUI, 2002) e um classificador K vizinhos mais próximos.


\section{Trabalhos Relacionados}
\label{relacionados}
