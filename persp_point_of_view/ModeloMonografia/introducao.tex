\chapter{Introdução}

%\section{Motivação}

A busca por opiniões sempre desempenhou um papel importante na geração de novas escolhas. Antes de optar por assistir a um filme, é comum ler críticas a seu respeito ou considerar os comentários de outras pessoas; antes de comprar um produto, muitas vezes procuramos relatos sobre a satisfação de outros consumidores. Com a disseminação da Web e da Internet, a geração de opiniões com impacto, sobre os mais diversos assuntos, foi finalmente democratizada: não é mais preciso, por exemplo, ser um especialista em Economia ou Ciência Política para manter um \emph{blog} convincente sobre algum candidato às eleições.

Neste contexto, a busca por opiniões e comentários em \emph{sites}, \emph{blogs}, fóruns e redes sociais também se popularizou, passando a fazer parte do cotidiano dos consumidores \emph{online}. Uma pesquisa feita nos Estados Unidos revelou que entre 73\% e 87\% dos leitores de resenhas de serviços online, como críticas de restaurantes e albergues, sentem-se fortemente influenciados a consumi-los ou não a depender das opiniões contidas nessas resenhas \cite{pesquisa-eua}. Diante da relevância que opiniões têm na geração de decisões e no processo de consumo, estudos com o intuito de extraí-las da Web e interpretá-las automaticamente tornaram-se mais frequentes na área de Ciência da Computação. Juntos, esses estudos compõem o que ficou conhecido como \textbf{Análise de Sentimento} ou \textbf{Mineração de Opinião}, termos utilizados como sinônimos \cite{omsa} \cite{bingliu}.

De acordo com a \emph{survey} de Pang e Lee, referência mais completa para o estudo da área encontrada até o momento, a Mineração de Opinião envolve o emprego de diversas técnicas computacionais com o intuito de explorar algum dos tópicos abaixo \cite{omsa}:

\begin{enumerate}
    \item \textbf{Polaridade de sentimento ou graus de polaridade -} Dado um documento opinativo, para o qual se assume que as opiniões se referem basicamente a um único assunto, classifique-o como positivo ou negativo em relação a esse assunto (polaridades opostas) ou localize-o no espectro estabelecido entre duas polaridades opostas;
    \item \textbf{Detecção de subjetividade e identificação de opinião -} Dado um documento, detecte se ele é subjetivo ou não, constituindo-se de fatos ou opiniões, ou que partes dele são subjetivas;
    \item \textbf{Análise de tópico-sentimento -} Dado um documento opinativo, assume-se que suas opiniões podem se referir a tópicos diferentes, e deve-se identificar quais opiniões interagem com quais tópicos;
   \item \textbf{Pontos de vista ou perspectivas -} Dado um documento opinativo, que apresente uma perspectiva sobre um tema (um ponto de vista, uma orientação ideológica, um posicionamento), em vez de um sentimento polarizado sobre um único assunto, classifique-o de acordo com essa perspectiva;
   \item \textbf{Outras informações não-factuais -} Dado um documento com caráter emotivo/sentimental, identifique que tipos de humor o permeiam e/ou classifique-o de acordo com as emoções encontradas.

\end{enumerate}


%No começo dos anos 2000, os termos \emph{Opinion Mining} e \emph{Sentiment Analysis} são empregados pela primeira vez, referindo-se ao emprego de técnicas computacionais para identificação e avaliação de opiniões e sentimentos em textos \cite{panglee}. Esses termos têm funcionado como sinônimos \cite{bing}, aparecendo paralelamente na literatura. De acordo com \cite{panglee}, o termo \emph{Sentiment Analysis} é empregado pela primeira vez em 2001, nos artigos \textbf{Como citar referências citadas em outro trabalho?}; o termo \emph{Opinion Mining}, em 2003, no artigo \textbf{mesma duvida}. Nesta monografia, os dois termos serão utilizados de forma intercambiável.


A ideia de utilizar metodologias computacionais para identificar e analisar opiniões é muito anterior à popularização da Web \textbf{Citar artigos do fim da década de 60 e começo de 70 que provam isso}. Motivos: pouco dado, IR e ML imaturas. Explicar os 3 e como se relacionam com Natural Language Processing.
%\textbf{POPULARIDADE DAS ÁREAS DE ML E IR, SURGIMENTO DE DATASETS DISPONIVEIS POR CONTA DA POPULARIDADE DA INTERNET...}.

%One of the main reasons for the lack of study on opinions is that there was little opinionated text before
%the World Wide Web. Before the Web, when an individual needs to make a decision, he/she typically
%asks for opinions from friends and families. 

%Falar das principais tasks da área.

\section{Proposta}

Falar de Mineração de Perspectiva. Definir todos os termos correlatos utilizados, fechar os problemas da área e explicar como isso se diferencia de Opinion Mining clássica, que é basicamente Análise de Polaridade.

%\textbf{explicar que aqui muitas vezes é mais interessante entender posicionamentos globais do que a opinião sobre um único assunto, tópico, etc, e que tb há pesquisas na área de opinion mining que tratam disso; mostrar onde diferem, os problemas} Especificamente no que diz respeito à política, uma pesquisa realizada nos Estados Unidos \cite{rainiehorrigan}, com mais de 2500 usuários de Internet que compartilharam informação sobre as eleições de 2006 \emph{online}, concluiu o seguinte:

%\begin{enumerate}
 % \item Para 28\% deles, um dos principais motivos para realizar esta troca de informações \emph{online} era compreender as perspectivas de suas próprias comunidades; 
  %\item 34\% deles afirmaram que um dos principais motivos para realizar esta troca de informações \emph{online} era compreender as perspectivas exteriores à suas comunidades; 
   %\item 27\% deles pesquisaram avaliações políticas de organizações externas \emph{online};
%  \item 28\% deles afirmaram que a maioria dos \emph{sites} de política que frequentam compartilham com seus pontos de vista;
 % \item 29\% deles afirmaram que a maioria dos \emph{sites} de política que frequentam apresentam pontos de vista que divergem dos seus - indicação de que muitas pessoas não estão buscando apenas uma validação de opiniões pré-existentes\cite{panglee};
  %\item 8\% deles publicam seus próprios comentários políticos \emph{online}.
%\end{enumerate}  

%\textbf{Concluir fechando mais o escopo para perspectivas; tentar motivar um pouco para eleições, MAAAAAS falar brevemente de trabalhos que não necessariamente tem a ver com política e que tb tentam capturar perspectivas/pontos de vista}.

%Falar que as bases abrangentes serão discutidas, mas essa mono fecha o escopo em viewpoints, blablabla. Falar que se pretende fazer uma revisão comparativa dos trabalhos, mostrar alguma coisa prática na área com as decisões de projeto devidamente justificadas.

\section{Estrutura da Monografia}

Falar da metodologia de busca dos artigos 
%Modelo de monografia usando as normas ABNT (Associação Brasileira de Normas
%%Técnicas) \sigla{ABNT}{Associação Brasileira de Normas Técnicas}
%e adaptação personalizada 
%do padrão do Departamento de Ciência da Computação (DCC) da Universidade
%Federal da Bahia (UFBA).
%\sigla{DCC}{Departamento de Ciência da Computação}
%\sigla{UFBA}{Universidade Federal da Bahia}
%Fontes latex cedidos pela ABNT e disponibilizados por 
%Maurício Vieira. (Valeu Maurix!). Adaptado por Abelmon Bastos por solicitação
%da \profa\ Débora Abdalla para o semestre 2005.1.
%Adaptado por Rodrigo Rocha por solicitação da \profa\ Débora Abdalla no fim
%do semestre 2007.1.

% use sua propria estrutura

%Problema etc

%Objetivo etc

%Resultados esperados etc

%Estrutura/Organização etc
