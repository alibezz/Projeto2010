% Capa com Brasão

\begin{titlepage}
   \begin{center}
      %logotipo
               \includegraphics{brasaoUFBA} \\
  %\vspace{0.7in}
              \centering{ 
        \bf{
        \LARGE{
    \uppercase{UNIVERSIDADE FEDERAL DA BAHIA} \\
        }
        \Large {
                    \uppercase{INSTITUTO DE MATEMÁTICA} \\
        }
                   \large {
                       \uppercase{DEPARTAMENTO DE CIÊNCIA DA COMPUTAÇÃO} \\
                  }
              } }
   \end{center}
   \vfill
   \begin{center}
       \bf{
       \large{\uppercase{\meunome}  \\  }
       }
   \end{center}
   \vspace{0.2in}
   \begin{center}
       \bf{
         \LARGE{ \uppercase{\meutitulo} } \\
         \Large{ \uppercase{\meusubtitulo} }
         \obs{\\ \Large{Esta versão da monografia contém comentários do autor.
          Para removê-los, redefina o comando LaTeX \texttt{obs}.}}
       }
   \end{center}

   \vfill
   \hspace{\stretch{1}}
   \vfill
   \begin{center}
      \normalsize{
          Salvador \\
          \meuano
       }
   \end{center}

\end{titlepage}

%comando abaixo cria uma capa redundante, mas como a capa com brasão foi 
% feita 'manualmente', não faz sentido usar este comando:
%\capa

