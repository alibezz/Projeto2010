
\chapter{Relação entre as características dos \emph{datasets} e as metodologias utilizadas}
\label{cap:caracsdatasets}

\textbf{PRECISA PASSAR POR REFORMULAÇÃO. Está provavelmente no lugar errado.}
Os \emph{datasets} estudados nesse projeto são oriundos de fontes diversas, incluindo \emph{blogs} \cite{jiang-argamon} \cite{durant-smith}, matérias jornalísticas \cite{grefenstette-et-al} \cite{schimmelfing-baldwin}, artigos escritos por especialistas \cite{lin-et-al2006} \cite{efrom}, discussões \emph{online} \cite{somasundaran} \cite{wiebe08} e debates políticos \cite{hirst-et-al} \cite{thomas-pang-lee}. Os assuntos discutidos também são bastante variados, incluindo tópicos relativamente abstratos, como a discussão da pena de morte \cite{greeneTESE}, e outros mais objetivos, como possíveis \emph{designs} para um controle remoto \cite{somasundaranGRAPH} \cite{wiebe08}. As linguagens empregadas nos documentos diferem bastante de um trabalho para outro, variando tanto na informalidade dos termos e construções empregadas quanto no teor opinativo das colocações \textbf{sigo citando?}. Outra característica importante, que distingue um estudo de outro, envolve a língua - ou línguas - nas quais os documentos se encontram. \textbf{Ler um pouco sobre isso para amadurecer este ponto} Por fim, o tamanho dos textos analisados, que varia de algumas sentenças a vários parágrafos, bem como o nível de engajamento de seus autores com as perspectivas defendidas, indica uma Web muito plural no que diz respeito aos tipos de conteúdo \emph{online}. 

Nos trabalhos estudados para este projeto, percebeu-se que as características inerentes a cada \emph{dataset} pouco interferem na decisão dos métodos utilizados na mineração das perspectivas dos documentos. No decorrer deste capítulo, a forte relação que existe entre essas características e a escolha das metodologias será discutida, justificando parcialmente os resultados ruins encontrados em alguns artigos. Adicionalmente, através de experimentos em \emph{datasets} referenciados nesses estudos, ou coletados \emph{online}, este capítulo apontará possibilidades metodológicas que podem conduzir a melhorias nos resultados analisados. \textbf{Devo enfatizar a originalidade disso aqui? Acho q n, né? Fica na problematização.} O capítulo está estruturado da seguinte forma: \textbf{blablabla}. Por fim, na \textbf{Seção Y}, algumas combinações de características comuns em documentos da Web, como alto teor de linguagem opinativa em debates informais \emph{online} \cite{somasundaran}, serão analisadas conjuntamente.

